\section{Literature Review} \label{sec:literature_review}

\cite*{02_rp} suggest that arrhythmia is one of the most common symptoms in patients with COVID19. {\responsemod Arrhythmia was found in 7\% of all Wuhan COVID-19 cases and 14.8\% of patients with poor outcomes.} \cite*{18_rp} state that 17\% of patients hospitalized in China were diagnosed with arrhythmia. \citeauthor{18_rp} conducted a review of 10 eligible studies (5,193 patients) for analysis and found that atrial arrhythmia was present in 9.2\% of cases. A review by \cite*{15_rp} of 17 studies with 5,815 patients showed that arrhythmia was detected in 9.3\% of COVID-19 cases. \cite*{25_rp} suggest that only 8\% of patients with arrhythmia had prior cardiovascular conditions. {\responsemod \citeauthor{25_rp} also mentioned that 56\% of patients showed symptoms after the COVID-19 infection.}

\cite*{24_rp} used an ensemble classifier to detect the anomalies in the ECG signal. This approach, which combines multiple classifiers for prediction, has proven effective because the accuracy of the ensemble classifier is significantly higher than that of a single classifier. A few authors used this approach to improve the prediction accuracy of supervised learning models. \cite*{10_rp} used the maximal overlap wavelet packet transform ensemble with a neural network and achieved satisfactory results. \cite*{20_rp} used the ensemble approach to predict the results of the students. \citeauthor{20_rp} state the model was able to predict the correct results even with a small amount of training data. \cite*{16_rp} compared the FLINK-based iForest ensembled algorithms against the sklearn-iForset and other algorithms. \citeauthor{16_rp} concluded that the Flink-iForest algorithm showed better performance than off-the-shelf algorithms. \cite*{11_rp} used the AutoML algorithm and tools on data streams. \citeauthor{11_rp} concluded that the default classifiers can be used with AutoML tools for accurate prediction. With AutoML tools, prediction systems can be automated.

\cite*{ref_paper_m1} used machine learning for the prediction of end-of-semester results. \cite*{ref_paper_m1} used SVM, KNN, and DT models. \cite*{ref_paper_m1} concluded that the machine learning system performed satisfactorily, with SVM achieving up to 78\% accuracy.

\cite*{23_rp} extracted appropriate features for the detection of epileptic seizures. \citeauthor{23_rp} preprocessed data and used ML algorithms on the data. \citeauthor{23_rp} concluded that the supervised learning models showed more effectiveness than the unsupervised learning models. \cite*{12_rp} used a commercial classifier for the detection of arrhythmia. \citeauthor{12_rp} used ECG signals from patients and applied a custom SVM classifier. \citeauthor{12_rp} concluded that the algorithm was a successful and efficient {\responsemod detector} of arrhythmia.

\cite*{ref_paper_m2} used supervised learning algorithms for the early detection of heart disease and diabetes disease. \cite*{ref_paper_m2} concluded that the model performed satisfactorily.

\cite*{09_rp} used neural networks to process raw ECG signals and make predictions. {\responsemod While \cite*{22_rp} used small neural networks for efficient recognition processes, both studies concluded that artificial neural networks are extremely efficient and accurate in the prediction of anomalies.}

\cite*{06_rp} showed that the LDA classifier can outperform the SVM classifier in low-performance environments and lightweight systems. The self-learning algorithm makes the system more dynamic and adaptable to incoming signals. \cite*{14_rp} used adaptive fuzzy algorithms to classify ECG signals. \citeauthor{14_rp} stated that the algorithm showed satisfactory results, but it requires {\responsemod prior} classification patterns results. \cite*{ref_paper_self_rpa} suggest that the RPA system can be used in these systems for easier integration of machine learning with dynamic data. \cite*{21_rp} used ECG signals of COVID-19 patients for patient monitoring. \citeauthor{21_rp} used LSTM, SVM, and MLP algorithms to monitor data. \citeauthor{21_rp} suggest that machine learning with robotics can provide better results.

\cite*{07_rp} used a multi perceptron neural network for stroke predictions. The neural network showed high accuracy. \citeauthor{07_rp} were able to achieve up to 78\% accuracy. \citeauthor{07_rp} suggest that the model can produce better results with a larger training dataset. \cite*{05_rp} used artificial intelligence to detect heart disease. \citeauthor{05_rp} concluded that the algorithms achieved up to 83\% accuracy. \citeauthor{05_rp} also concluded that the system was able to comply with the HIPAA regulations.

\cite*{ref_paper_m3} used machine learning algorithms to predict crop growth rates. {\responsemod \citeauthor{ref_paper_m3} were able to get good insights into the field.} \citeauthor{ref_paper_m3} concluded that the use of machine learning will result in minimizing complexity and increasing yield in farming.

\cite*{01_rp} used a two-year dataset collected by Glumo Lake and used their expertise to train and select models. A mixed approach of data-driven and knowledge-driven modeling is used for the success of the application. \cite*{13_rp} used loo rate and stop criteria for model selection. \citeauthor{13_rp} investigated eight different issues and found that a larger loo rate was more desirable. \citeauthor{13_rp} also suggested that modeling difficulties can only be found by careful numerical calculations.

\cite*{17_rp} used a novel kernel to get the dataset description. \citeauthor{17_rp} concluded that this approach led to the discovery of invisible models. \citeauthor{17_rp} also state that this approach reduces the amount of human interaction. \cite*{04_rp} created a new model using the glmulti package. These models are unique and flexible. The model is automatically optimized to provide a multi-model interface. This approach allows you to quickly explore a large set of models for selection purposes. \cite*{08_rp} optimized parameters with a genetic algorithm. \citeauthor{08_rp} successfully used a genetic algorithm to reduce uncertainty in the prediction results. These methods can be used for the automated model selection system.
