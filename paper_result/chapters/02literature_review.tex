\section{Literature Review} \label{sec:literature_review}
Arrhythmia is one of the most common symptoms in patients with COVID19 \cite{babapoor2020arrhythmia}. Arrhythmia was found in 7\% of all 19 Wuhan COVID cases and 14.8\% of patients with poor outcomes. Data from 17 studies of 5815 patients suggest that up to 9.3\% of confirmed patients detected arrhythmia \cite{mulia2021atrial, liu2020clinical}. Up to 94.4\% of all deceased patients had arrhythmia and 95.8\% of patients with severe infections had arrhythmia \cite{beri2020cardiac, ren2020clinical}. According to the studies \cite{babapoor2020arrhythmia, liu2020clinical, yarmohammadi2021frequency}, only 8\% of patients with arrhythmia had previously had cardiovascular disease, but 56\% of the symptoms of arrhythmia recurred after having COVID19 disease.

In the work \cite{sun2020multi}, the author used an ensemble classifier to detect anomalies in the ECG signal. This approach, which combines multiple classifiers for prediction, has proven effective because the accuracy of the ensemble classifier is significantly higher than that of a single classifier. This approach has also been used by other researchers to improve the prediction accuracy of supervised learning models \cite{huang2020accurate, rajak2020applying, liu2020parallel}. This study also showed high accuracy and validation scores for anomaly detection \cite{huang2020accurate, liu2020parallel}, while one study made accurate predictions without prior information about \cite{rajak2020applying} in the same class. Was shown to be obtained. You can use the default classifier to automatically get an accurate and accurate prediction of the incoming data stream \cite{imbrea2021automated}.

To detect the presence of a patient's epileptic seizure, the authors extracted appropriate features and preprocessed the data by applying supervised and unsupervised machine learning classifiers to the patient. Predictions from test data show that supervised learning models are more effective than unstructured learning models \cite{siddiqui2020review}. The default classifier can be automated for accurate and accurate predictions \cite{imbrea2021automated}. The authors of the \cite{jha2020cardiac} study also showed that a commercially available classifier SVM is very efficient in detecting arrhythmia in the ECG signal.

With the ensemble classifier approach, anomalies can be detected by combining multiple classifiers for prediction. The ensemble classifiers can be more effective and accurate than individual supervised learning models \cite{sun2020multi, liu2020parallel, huang2020accurate}. The author of paper \cite{rajak2020applying} discovered that accurate predictions can be obtained with limited or no prior information about past target values of the same class.

Artificial neural networks are also very effective and accurate in detecting arrhythmia from ECG signals. Neural networks can process raw ECG signals and make predictions \cite{hannun2019cardiologist}. Small neural networks are also very efficient in the recognition process \cite{sannino2018deep}. Neural network predictions are relatively accurate and accurate than known classifiers. Lightweight classifiers such as the LDA classifier showed higher accuracy than SVM classifiers for low energy systems \cite{chen2013design}. The self-adaptive learning algorithm is also efficient for low energy lightweight systems \cite{lei2007afc, owis2002study}. The self-learning algorithm makes the system more dynamic and adapts to incoming signals. This dynamic system takes very little time for identification tasks, but classification tasks are more difficult and very time-consuming.

In the work \cite{natasa2008}, the author uses a two-year dataset collected by Glumo Lake and uses his expertise to train and select models. A mixed approach of data-driven and knowledge-driven modeling is used for the success of the application. As the author of the article \cite{lee2000automatic}, he used loo rate and stop criteria for model selection. The author investigated eight different issues and found that a larger loo rate was more desirable. The authors also suggested that modeling difficulties can only be found by careful numerical calculations.

In the article \cite{malkomes2016bayesian}, a novel kernel is used to get the dataset description. This approach leads to the discovery of invisible models with minimal human interaction. The author of the article \cite{JSSv034i12} created a new model using the glmulti package. These models are unique and flexible. The model is automatically optimized to provide a multi-model interface. This approach allows you to quickly explore a large set of models for selection purposes.