\section{Introduction} \label{sec:introduction}

COVID-19 has been widespread in recent years. It targets the human respiratory system, causing severe respiratory issues. Depending on a person's condition and the prevalence of comorbidities, this disease can be fatal. {\responsemod COVID-19 disease frequently causes cardiovascular comorbidities.} Cardiovascular comorbidities are also problematic to diagnose in the absence of suitable equipment. Checking for arrhythmia in patients is one approach to detection. Arrhythmia is the irregular beating of the heart. Arrhythmia is detected by examining ECG signals. Because COVID-19 has put a strain on the medical personnel, detection takes longer than usual. Increased Internet connectivity has led to the use of machine learning and artificial intelligence (AI) for service in a variety of sectors. This increases research in the field of machine learning and has an impact on machine learning in a variety of domains. One of them is the medical and healthcare {\responsemod industries}. Machine learning is used to detect and categorize viruses and other microorganisms in patients. In medical applications, machine learning algorithms have already been shown to be quite useful.

The machine learning system may be used to scan these ECG signals and detect them. These signs may be detected considerably faster and more efficiently using supervised learning techniques. In such exact classification problems, supervised algorithms have previously been demonstrated to be quicker than unsupervised techniques. Once taught, this algorithm may also be utilized to make future predictions.

There are several supervised algorithms {\responsemod available}, allowing us to select the best method for our purposes. This phase can be automated in the case of the general population. A few methods may be pre-programmed into the system, and the computer can then train and pick the best model for the supplied dataset. This will free up medical personnel to focus on patient care and problem-solving.
