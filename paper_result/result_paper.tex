%% 
%% Copyright 2019-2021 Elsevier Ltd
%% 
%% This file is part of the 'CAS Bundle'.
%% --------------------------------------
%% 
%% It may be distributed under the conditions of the LaTeX Project Public
%% License, either version 1.2 of this license or (at your option) any
%% later version.  The latest version of this license is in
%%    http://www.latex-project.org/lppl.txt
%% and version 1.2 or later is part of all distributions of LaTeX
%% version 1999/12/01 or later.
%% 
%% The list of all files belonging to the 'CAS Bundle' is
%% given in the file `manifest.txt'.
%% 
%% Template article for cas-dc documentclass for 
%% double column output.

\documentclass[a4paper,fleqn]{cas-dc}
\usepackage{cleveref}
\usepackage{placeins}
\usepackage{afterpage}


% If the frontmatter runs over more than one page
% use the longmktitle option.

%\documentclass[a4paper,fleqn,longmktitle]{cas-dc}

% \usepackage[numbers]{natbib}
\usepackage[authoryear]{natbib}
% \usepackage[authoryear,longnamesfirst]{natbib}

%%%Author macros
\def\tsc#1{\csdef{#1}{\textsc{\lowercase{#1}}\xspace}}
\tsc{WGM}
\tsc{QE}
%%%

% Uncomment and use as if needed
%\newtheorem{theorem}{Theorem}
%\newtheorem{lemma}[theorem]{Lemma}
%\newdefinition{rmk}{Remark}
%\newproof{pf}{Proof}
%\newproof{pot}{Proof of Theorem \ref{thm}}

\renewcommand{\dblfloatpagefraction}{0.4}

\begin{document}
\let\WriteBookmarks\relax
\def\floatpagepagefraction{1}
\def\textpagefraction{.001}

% Short title
% \shorttitle{<short title of the paper for running head>}    
\shorttitle{Weightage-based Supervised Learning System.}

% Short author
% \shortauthors{<short author list for running head>}  
\shortauthors{Ketkar Y., Gawade S.}

% Main title of the paper
\title [mode = title]{Detection of Arrhythmia Using Weightage-based Supervised Learning System for COVID-19.}

\let\printorcid\relax

% Title footnote mark
% eg: \tnotemark[1]
% \tnotemark[<tnote number>] 

% Title footnote 1.
% eg: \tnotetext[1]{Title footnote text}
% \tnotetext[<tnote number>]{<tnote text>} 

% First author
%
% Options: Use if required
% eg: \author[1,3]{Author Name}[type=editor,
%       style=chinese,
%       auid=000,
%       bioid=1,
%       prefix=Sir,
%       orcid=0000-0000-0000-0000,
%       facebook=<facebook id>,
%       twitter=<twitter id>,
%       linkedin=<linkedin id>,
%       gplus=<gplus id>]

\author[1]{Yashodhan Ketkar}

% Corresponding author indication
% \cormark[<corr mark no>]

% Footnote of the first author
% \fnmark[<footnote mark no>]

% Email id of the first author
\ead{ketkaryapr19me@student.mes.ac.in}

% URL of the first author
% \ead[url]{<URL>}

% Credit authorship
% eg: \credit{Conceptualization of this study, Methodology, Software}
% \credit{<Credit authorship details>}

% Address/affiliation
\affiliation[1]{organization={Department of Information Technology Engineering, Pillai College of Engineering},
            % addressline={}, 
            city={Panvel},
%          citysep={}, % Uncomment if no comma needed between city and postcode
            postcode={410206}, 
            state={Maharashtra},
            country={India}}

\author[2]{Sushopti Gawade}

% Footnote of the second author
% \fnmark[2]

% Email id of the second author
\ead{sgawade@mes.ac.in}

% URL of the second author
% \ead[url]{}

% Credit authorship
% \credit{}

% Address/affiliation
\affiliation[2]{organization={Department of Computer Engineering, Pillai College of Engineering},
            % addressline={}, 
            city={Panvel},
%          citysep={}, % Uncomment if no comma needed between city and postcode
            postcode={410206}, 
            state={Maharashtra},
            country={India}}

% Corresponding author text
% \cortext[2]{Corresponding author}

% Footnote text
% \fntext[1]{}

% For a title note without a number/mark
%\nonumnote{}

% Here goes the abstract
\begin{abstract}
COVID-19 disease became a global pandemic in the last few years. This disease was highly contagious, and it quickly spread throughout several countries. Its infection can lead to severe implications in its victims, including cardiovascular issues. This complication develops in some people with a history of cardiovascular illness, whereas it emerges in others after COVID-19 infection. Cardiovascular problems are the primary cause of mortality in COVID-19 patients and are used to predict disease prognosis. Identifying arrhythmia from abnormalities in patient ECG signals is one approach to the detection of cardiovascular disorders. This is a laborious and time-consuming procedure that can be automated. The proposed method selects the most suitable model for this task. The selection is done through the weightage generated from the user's requirements. The proposed method uses supervised learning to identify abnormalities in ECG waves. The models provided by the selection system during tests were able to meet user requirements. The models achieved up to 97\% accuracy and 97\% precision in predictive tasks.
\end{abstract}

% Use if graphical abstract is present
%\begin{graphicalabstract}
%\includegraphics{}
%\end{graphicalabstract}

% Research highlights
% \begin{highlights}
% \item a
% \item a
% \item a
% \end{highlights}

% Keywords
% Each keyword is seperated by \sep
\begin{keywords}
    Arrhythmia Detection \sep
    Automated Model Generation \sep
    Automated Model Training \sep
    Machine Learnign In Healthcare \sep
    Supervised Learning Algoirthm \sep
    Weightage based Model Selection \sep
\end{keywords}

\tolerance 9999
\maketitle


% Main text
% \section{Introduction}\label{sec:introduction}
\section{Introduction}\label{sec:introduciton}

In recent years, machine learning has become very popular. Earlier, machine learning was limited to the research field. But quite recently, it has been used in various fields. This is partially due to the increasing availability of computers and growth in computing power.

The researchers are using machine learning in various fields. One of such fields is solar energy production. In this field, the prediction of solar radiation available per day can be very significant. \cite*{ref_paper_7} employed a few machine learning algorithms for the prediction of the amount of solar radiation received in a day. \citeauthor{ref_paper_7} used natural properties such as weather, time, etc as features for the machine learning algorithms. \citeauthor{ref_paper_7} concluded that supervised learning models gave satisfactory results in predicting the amount of solar radiation. \citeauthor{ref_paper_7} concluded that the ANN model showed great potential for such specialized tasks.

In the field of chemistry, there is a wide range of factors that need to be considered. \cite*{ref_paper_10} used machine learning in the field of chemistry. \citeauthor{ref_paper_10} suggest that the models performed better than conventional statistical methods. \citeauthor{ref_paper_10} concluded that more research is necessary before implementation.

Machine learning can be applied in network security for intrusion detection. \cite*{ref_paper_21} state that the supervised learning methods perform satisfactorily for intrusion detection. \citeauthor{ref_paper_21} also remark that the supervised learning methods are limited to conventional problems.

\cite*{ref_paper_36} in thier study used machine learning models to solve real-life problems. \citeauthor{ref_paper_36} concluded that machine learning plays an important role in good decision-making. \citeauthor{ref_paper_36} also remark that machine learning can be used in various fields.

\cite*{ref_paper_24} note a lack of information and meaningful research in the medical field. \citeauthor{ref_paper_24} suggest that more support, subject-specific scope, and accuracy are important factors for the higher impact of machine learning in the medical field. \citeauthor{ref_paper_24} also suggest that the ideal system should be able to take multiple data types for training.

\cite*{ref_paper_29} used machine learning with a combination of IoT for smart cities. \citeauthor{ref_paper_29} concludes that machine learning showed promising results for smart cities. \citeauthor{ref_paper_29} remark that machine learning was able to handle the high volume of data generated by sensors. \cite*{ref_paper_12} in thier paper suggests that machine learning can handle downstream tasks efficiently.

This suggests that even with the immense popularity and accessibility of machine learning, it is still underutilized by the general population. This can be attributed to lack of knowledge and skills. To solve this problem we are proposing a system with minimum user interaction. The proposed system can be operated by users without prior knowledge of machine learning. This system allows them to train their machine learning algorithms.

The proposed system can be deployed on a local network. The system can feed data manually or automatically according to user needs and policies. The local deployment also limits external access and reduces the influence of external factors on data.

\section{Literature Review} \label{sec:literature_review}

\cite*{02_rp} suggest that arrhythmia is one of the most common symptoms in patients with COVID19. {\responsemod Arrhythmia was found in 7\% of all Wuhan COVID-19 cases and 14.8\% of patients with poor outcomes.} \cite*{18_rp} state that 17\% of patients hospitalized in China were diagnosed with arrhythmia. \citeauthor{18_rp} conducted a review of 10 eligible studies (5,193 patients) for analysis and found that atrial arrhythmia was present in 9.2\% of cases. A review by \cite*{15_rp} of 17 studies with 5,815 patients showed that arrhythmia was detected in 9.3\% of COVID-19 cases. \cite*{25_rp} suggest that only 8\% of patients with arrhythmia had prior cardiovascular conditions. {\responsemod \citeauthor{25_rp} also mentioned that 56\% of patients showed symptoms after the COVID-19 infection.}

\cite*{24_rp} used an ensemble classifier to detect the anomalies in the ECG signal. This approach, which combines multiple classifiers for prediction, has proven effective because the accuracy of the ensemble classifier is significantly higher than that of a single classifier. A few authors used this approach to improve the prediction accuracy of supervised learning models. \cite*{10_rp} used the maximal overlap wavelet packet transform ensemble with a neural network and achieved satisfactory results. \cite*{20_rp} used the ensemble approach to predict the results of the students. \citeauthor{20_rp} state the model was able to predict the correct results even with a small amount of training data. \cite*{16_rp} compared the FLINK-based iForest ensembled algorithms against the sklearn-iForset and other algorithms. \citeauthor{16_rp} concluded that the Flink-iForest algorithm showed better performance than off-the-shelf algorithms. \cite*{11_rp} used the AutoML algorithm and tools on data streams. \citeauthor{11_rp} concluded that the default classifiers can be used with AutoML tools for accurate prediction. With AutoML tools, prediction systems can be automated.

\cite*{ref_paper_m1} used machine learning for the prediction of end-of-semester results. \cite*{ref_paper_m1} used SVM, KNN, and DT models. \cite*{ref_paper_m1} concluded that the machine learning system performed satisfactorily, with SVM achieving up to 78\% accuracy.

\cite*{23_rp} extracted appropriate features for the detection of epileptic seizures. \citeauthor{23_rp} preprocessed data and used ML algorithms on the data. \citeauthor{23_rp} concluded that the supervised learning models showed more effectiveness than the unsupervised learning models. \cite*{12_rp} used a commercial classifier for the detection of arrhythmia. \citeauthor{12_rp} used ECG signals from patients and applied a custom SVM classifier. \citeauthor{12_rp} concluded that the algorithm was a successful and efficient {\responsemod detector} of arrhythmia.

\cite*{ref_paper_m2} used supervised learning algorithms for the early detection of heart disease and diabetes disease. \cite*{ref_paper_m2} concluded that the model performed satisfactorily.

\cite*{09_rp} used neural networks to process raw ECG signals and make predictions. {\responsemod While \cite*{22_rp} used small neural networks for efficient recognition processes, both studies concluded that artificial neural networks are extremely efficient and accurate in the prediction of anomalies.}

\cite*{06_rp} showed that the LDA classifier can outperform the SVM classifier in low-performance environments and lightweight systems. The self-learning algorithm makes the system more dynamic and adaptable to incoming signals. \cite*{14_rp} used adaptive fuzzy algorithms to classify ECG signals. \citeauthor{14_rp} stated that the algorithm showed satisfactory results, but it requires {\responsemod prior} classification patterns results. \cite*{ref_paper_self_rpa} suggest that the RPA system can be used in these systems for easier integration of machine learning with dynamic data. \cite*{21_rp} used ECG signals of COVID-19 patients for patient monitoring. \citeauthor{21_rp} used LSTM, SVM, and MLP algorithms to monitor data. \citeauthor{21_rp} suggest that machine learning with robotics can provide better results.

\cite*{07_rp} used a multi perceptron neural network for stroke predictions. The neural network showed high accuracy. \citeauthor{07_rp} were able to achieve up to 78\% accuracy. \citeauthor{07_rp} suggest that the model can produce better results with a larger training dataset. \cite*{05_rp} used artificial intelligence to detect heart disease. \citeauthor{05_rp} concluded that the algorithms achieved up to 83\% accuracy. \citeauthor{05_rp} also concluded that the system was able to comply with the HIPAA regulations.

\cite*{ref_paper_m3} used machine learning algorithms to predict crop growth rates. {\responsemod \citeauthor{ref_paper_m3} were able to get good insights into the field.} \citeauthor{ref_paper_m3} concluded that the use of machine learning will result in minimizing complexity and increasing yield in farming.

\cite*{01_rp} used a two-year dataset collected by Glumo Lake and used their expertise to train and select models. A mixed approach of data-driven and knowledge-driven modeling is used for the success of the application. \cite*{13_rp} used loo rate and stop criteria for model selection. \citeauthor{13_rp} investigated eight different issues and found that a larger loo rate was more desirable. \citeauthor{13_rp} also suggested that modeling difficulties can only be found by careful numerical calculations.

\cite*{17_rp} used a novel kernel to get the dataset description. \citeauthor{17_rp} concluded that this approach led to the discovery of invisible models. \citeauthor{17_rp} also state that this approach reduces the amount of human interaction. \cite*{04_rp} created a new model using the glmulti package. These models are unique and flexible. The model is automatically optimized to provide a multi-model interface. This approach allows you to quickly explore a large set of models for selection purposes. \cite*{08_rp} optimized parameters with a genetic algorithm. \citeauthor{08_rp} successfully used a genetic algorithm to reduce uncertainty in the prediction results. These methods can be used for the automated model selection system.

\section{Dataset and Method} \label{sec:dataset_and_method}
\subsection{Method} \label{subsec:method}

\begin{figure}[ht]
    \centering
    \includegraphics[width=0.7\columnwidth]{media/design/Data_Flow_System.pdf}
    \caption{Training and Selection Process}
    \label{fig:data_flow_in_system}
\end{figure}

The figure \ref{fig:data_flow_in_system}, shows the basic architecture of the automated model training and selection system. The data is collected from the user and processed by the application. This data is stored as training and testing datasets.

\begin{figure}[ht]
    \centering
    \includegraphics[width=0.7\columnwidth]{media/design/Trainer.pdf}
    \caption{Training Process}
    \label{fig:training_process}
\end{figure}

Figure \ref{fig:training_process}, shows the training process. In this process, the training dataset is loaded into the system. Premade classification model templates are accessed by the system and trained with provided data. These trained models are stored by the system for the next step.

\begin{figure}[ht]
    \centering
    \includegraphics[width=0.7\columnwidth]{media/design/Selector.pdf}
    \caption{Selection Process}
    \label{fig:selection_process}
\end{figure}

The figure \ref{fig:selection_process}, shows the selection process. In this process, the testing dataset is used for the evaluation of the trained models. The trained models are ranked with respect to the performance evaluation. These ranks are used with the help of the tuning parameters to select the best-suited model. This model is stored as the best model for future classification.

\subsection{Dataset}
The ECG readings in this paper are obtained from the MIT-BIH arrhythmia database. This database is used for automated training and evaluation. This dataset was published in 1999 by MIT-BIH as an open-source database; it consists of training and testing datasets. This database is further divided into four equal parts for analysis. Each training set contains 21888 signals, and the testing set contains 5473 signals.
\section{Results And Testing} \label{sec:results_and_testing}

\subsection{Performance Evaluation} \label{subsec:performance_evaluation}

{\responsemod
    The system is designed to use supervised algorithms for model generation. We used K-Nearest Neighbors, Decision Tree, Multilayer Perceptron, Random Forest, and Support Vector Machine algorithms for the classification problem. Performance metrics used for evaluation were accuracy, F1, precision, recall, area under the ROC Curve, and total prediction time. The preference assigned to this metric was accuracy, which had the first preference. The F1 score had the second preference. The preferences for the product, recall, and ROC were set to auto. The training approach was selected with care. The system generated a weightage of 1.0, 0.8, 0.4, 0.4, 0.4, and 0.25.

    During the automated training and selection process models with the highest Vscore being selected as the most-suited models. The \Cref{tab:vscores_models} show the Vscore of the models. From this Vscores SVM classifier is selected as the best-suited model for datasets 2, 3, and 4. Whereas the RF classifier is selected as the most-suited model for dataset 1.
}

\Cref{fig:perfromance_results_dataset_1,fig:perfromance_results_dataset_2,fig:perfromance_results_dataset_3,fig:perfromance_results_dataset_4} show that models trained with an automated system produced satisfactory results. A few models, like KNN, RF, and SVM, performed better than other models (DT) at the cost of higher prediction time. Whereas MLP models produced good overall results with lower prediction times.
\Crefrange{tab:performance_of_models_trained_on_dataset_1}{tab:performance_of_models_trained_on_dataset_4} show the overall performance of the models on their respective datasets in detail.

\subsection{Performance Error Calculation}
For error calculation, the best models are tested against the training datasets of other models. \Cref{fig:perfromance_delta_knn,fig:perfromance_delta_dt,fig:perfromance_delta_mlp,fig:perfromance_delta_rf,fig:perfromance_delta_svm} show the average error introduced when models are tested against the training datasets of other models. This chart shows that KNN, MLP, and SVM models introduced minimum errors, whereas DT and RF models introduced large amounts of error. \Cref{fig:perfromance_delta_svm} also shows that the SVM model produced similar errors across all datasets. This smaller difference in error suggests that the SVM classifier can be used for classification tasks of similar nature. {\responsemod \Crefrange{tab:performance_k_nearest_neighbors_multi}{tab:performance_support_vector_machine_multi} show the performance of the models when tested on other datasets in detail.}

\section{Conclusion and Future work} \label{sec:conclusion_and_future_work}
In this paper, we present a novel system. The system provides the end user the ability to train the best-suited model for the problem. With the current COVID-19 pandemic, this system can be employed by healthcare professionals for the detection of anomalies such as arrhythmia in patients. The system is tested with the ECG MIT-BIH arrhythmia database. {\responsemod The models trained and selected by the system showed good classification performance. These models also performed satisfactorily against training datasets of other models, suggesting good general classification performance.}

Future work will involve the use of other freely available datasets to test the general classification performance of the system as well as testing the current system in a real-time environment. {\responsemod Future work will also focus on modifying the system to work with non-labeled databases by employing unsupervised learning methods.}

\afterpage{\clearpage
    % Best model Performance tabular
    \begin{table}[hbt]
        \caption{Performance of models trained on dataset 1} \label{tab:performance_of_models_trained_on_dataset_1}
        \begin{tabular*}{\tblwidth}{@{}LLLLLL@{}}
            \toprule
            Metric & KNN & DT & MLP & RF & \textbf{SVM} \\
            \midrule
            Accuracy & 96.72 & 94.46 & 96.89 & 97.33 & \textbf{97.40} \\
            F1 & 89.71 & 83.43 & 90.05 & 91.30 & \textbf{91.69} \\
            Precision & 92.53 & 81.86 & 94.71 & 97.95 & \textbf{96.43} \\
            Recall & 87.06 & 85.06 & 85.84 & 85.50 & \textbf{87.40} \\
            ROC & 92.84 & 90.68 & 92.45 & 92.57 & \textbf{93.38} \\
            Time(s) & 0.457 & 0.001 & 0.002 & 0.015 & \textbf{0.297} \\
            \bottomrule
        \end{tabular*}
    \end{table}

    \begin{table}[hbt]
        \caption{Performance of models trained on dataset 2} \label{tab:performance_of_models_trained_on_dataset_2}
        \begin{tabular*}{\tblwidth}{@{}LLLLLL@{}}
            \toprule
            Metric & KNN & DT & MLP & RF & \textbf{SVM} \\
            \midrule
            Accuracy & 96.83 & 95.04 & 96.69 & 97.09 & \textbf{97.46} \\
            F1 & 90.28 & 85.50 & 89.73 & 90.75 & \textbf{92.15} \\
            Precision & 94.25 & 84.90 & 94.73 & 98.60 & \textbf{96.79} \\
            Recall & 86.63 & 86.09 & 85.23 & 84.05 & \textbf{87.93} \\
            ROC & 92.77 & 91.48 & 92.13 & 91.90 & \textbf{93.66} \\
            Time(s) & 0.435 & 0.001 & 0.003 & 0.014 & \textbf{0.295} \\
            \bottomrule
        \end{tabular*}
    \end{table}

    \begin{table}[hbt]
        \caption{Performance of models trained on dataset 3} \label{tab:performance_of_models_trained_on_dataset_3}
        \begin{tabular*}{\tblwidth}{@{}LLLLLL@{}}
            \toprule
            Metric & KNN & DT & MLP & \textbf{RF} & SVM \\
            \midrule
            Accuracy & 97.07 & 94.64 & 96.41 & \textbf{97.22} & 97.44 \\
            F1 & 90.93 & 84.30 & 89.34 & \textbf{91.21} & 92.00 \\
            Precision & 95.82 & 83.81 & 90.13 & \textbf{98.37} & 97.81 \\
            Recall & 86.53 & 84.80 & 88.57 & \textbf{85.02} & 86.85 \\
            ROC & 92.87 & 90.73 & 93.29 & \textbf{92.36} & 93.22 \\
            Time(s) & 0.404 & 0.001 & 0.002 & \textbf{0.017} & 0.293 \\
            \bottomrule
        \end{tabular*}
    \end{table}

    \begin{table}[hbt]
        \caption{Performance of models trained on dataset 4} \label{tab:performance_of_models_trained_on_dataset_4}
        \begin{tabular*}{\tblwidth}{@{}LLLLLL@{}}
            \toprule
            Metric & KNN & DT & MLP & RF & \textbf{SVM} \\
            \midrule
            Accuracy & 96.84 & 94.99 & 95.34 & 96.88 & \textbf{97.15} \\
            F1 & 90.52 & 85.73 & 85.01 & 90.37 & \textbf{91.39} \\
            Precision & 95.71 & 85.91 & 97.83 & 98.65 & \textbf{97.41} \\
            Recall & 85.86 & 85.55 & 75.15 & 83.37 & \textbf{86.07} \\
            ROC & 92.52 & 91.28 & 87.40 & 91.56 & \textbf{92.79} \\
            Time(s) & 0.452 & 0.001 & 0.002 & 0.014 & \textbf{0.294} \\
            \bottomrule
        \end{tabular*}
    \end{table}

    % Decision Tree Cross-Performance
    \begin{table}[Ht!]
        \caption{Performance of Decision Tree model trained on dataset 1}\label{tab:performance_of_decision_tree_model_trained_on_dataset_1}
        \begin{tabular*}{\tblwidth}{@{}LLLLL@{}}
            \toprule
            Metric & Dataset 1 & Dataset 2 & Dataset 3 & Dataset 4 \\
            \midrule
            Accuracy & 0.98 & 0.94 & 0.94 & 0.94 \\
            F1 & 0.96 & 0.85 & 0.85 & 0.84 \\
            Precision & 0.95 & 0.83 & 0.84 & 0.83 \\
            Recall & 0.96 & 0.86 & 0.85 & 0.85 \\
            ROC & 0.97 & 0.91 & 0.91 & 0.90 \\
            \bottomrule
        \end{tabular*}
    \end{table}

    \begin{table}[Ht!]
        \caption{Performance of Decision Tree model trained on dataset 2}\label{tab:performance_of_decision_tree_model_trained_on_dataset_2}
        \begin{tabular*}{\tblwidth}{@{}LLLLL@{}}
            \toprule
            Metric & Dataset 1 & Dataset 2 & Dataset 3 & Dataset 4 \\
            \midrule
            Accuracy & 0.94 & 0.98 & 0.94 & 0.94 \\
            F1 & 0.84 & 0.96 & 0.84 & 0.85 \\
            Precision & 0.85 & 0.96 & 0.85 & 0.85 \\
            Recall & 0.82 & 0.96 & 0.84 & 0.84 \\
            ROC & 0.89 & 0.97 & 0.90 & 0.90 \\
            \bottomrule
        \end{tabular*}
    \end{table}

    \begin{table}[Ht!]
        \caption{Performance of Decision Tree model trained on dataset 3}\label{tab:performance_of_decision_tree_model_trained_on_dataset_3}
        \begin{tabular*}{\tblwidth}{@{}LLLLL@{}}
            \toprule
            Metric & Dataset 1 & Dataset 2 & Dataset 3 & Dataset 4 \\
            \midrule
            Accuracy & 0.94 & 0.94 & 0.98 & 0.94 \\
            F1 & 0.84 & 0.84 & 0.96 & 0.83 \\
            Precision & 0.84 & 0.83 & 0.95 & 0.83 \\
            Recall & 0.84 & 0.85 & 0.96 & 0.84 \\
            ROC & 0.90 & 0.90 & 0.97 & 0.90 \\
            \bottomrule
        \end{tabular*}
    \end{table}

    \begin{table}[Ht!]
        \caption{Performance of Decision Tree model trained on dataset 4}\label{tab:performance_of_decision_tree_model_trained_on_dataset_4}
        \begin{tabular*}{\tblwidth}{@{}LLLLL@{}}
            \toprule
            Metric & Dataset 1 & Dataset 2 & Dataset 3 & Dataset 4 \\
            \midrule
            Accuracy & 0.94 & 0.94 & 0.95 & 0.98 \\
            F1 & 0.85 & 0.85 & 0.85 & 0.96 \\
            Precision & 0.85 & 0.85 & 0.85 & 0.96 \\
            Recall & 0.85 & 0.84 & 0.85 & 0.96 \\
            ROC & 0.91 & 0.90 & 0.91 & 0.97 \\
            \bottomrule
        \end{tabular*}
    \end{table}

    % KNN Cross-Performance
    \begin{table}[Ht!]
        \caption{Performance of K-Nearest Neighbors model trained on dataset 1}\label{tab:performance_of_knn_model_trained_on_dataset_1}
        \begin{tabular*}{\tblwidth}{@{}LLLLL@{}}
            \toprule
            Metric & Dataset 1 & Dataset 2 & Dataset 3 & Dataset 4 \\
            \midrule
            Accuracy & 0.97 & 0.97 & 0.97 & 0.96 \\
            F1 & 0.93 & 0.91 & 0.90 & 0.90 \\
            Precision & 0.96 & 0.95 & 0.94 & 0.94 \\
            Recall & 0.90 & 0.87 & 0.87 & 0.87 \\
            ROC & 0.94 & 0.93 & 0.93 & 0.93 \\
            \bottomrule
        \end{tabular*}
    \end{table}

    \begin{table}[Ht!]
        \caption{Performance of K-Nearest Neighbors model trained on dataset 2}\label{tab:performance_of_knn_model_trained_on_dataset_2}
        \begin{tabular*}{\tblwidth}{@{}LLLLL@{}}
            \toprule
            Metric & Dataset 1 & Dataset 2 & Dataset 3 & Dataset 4 \\
            \midrule
            Accuracy & 0.96 & 0.97 & 0.96 & 0.96 \\
            F1 & 0.90 & 0.93 & 0.90 & 0.89 \\
            Precision & 0.94 & 0.96 & 0.94 & 0.94 \\
            Recall & 0.86 & 0.90 & 0.86 & 0.85 \\
            ROC & 0.92 & 0.94 & 0.92 & 0.92 \\
            \bottomrule
        \end{tabular*}
    \end{table}

    \begin{table}[Ht!]
        \caption{Performance of K-Nearest Neighbors model trained on dataset 3}\label{tab:performance_of_knn_model_trained_on_dataset_3}
        \begin{tabular*}{\tblwidth}{@{}LLLLL@{}}
            \toprule
            Metric & Dataset 1 & Dataset 2 & Dataset 3 & Dataset 4 \\
            \midrule
            Accuracy & 0.96 & 0.96 & 0.97 & 0.96 \\
            F1 & 0.90 & 0.90 & 0.93 & 0.90 \\
            Precision & 0.95 & 0.95 & 0.96 & 0.94 \\
            Recall & 0.86 & 0.86 & 0.89 & 0.86 \\
            ROC & 0.92 & 0.92 & 0.94 & 0.92 \\
            \bottomrule
        \end{tabular*}
    \end{table}

    \begin{table}[Ht!]
        \caption{Performance of K-Nearest Neighbors model trained on dataset 4}\label{tab:performance_of_knn_model_trained_on_dataset_4}
        \begin{tabular*}{\tblwidth}{@{}LLLLL@{}}
            \toprule
            Metric & Dataset 1 & Dataset 2 & Dataset 3 & Dataset 4 \\
            \midrule
            Accuracy & 0.96 & 0.96 & 0.96 & 0.97 \\
            F1 & 0.90 & 0.90 & 0.90 & 0.92 \\
            Precision & 0.94 & 0.95 & 0.94 & 0.96 \\
            Recall & 0.86 & 0.86 & 0.87 & 0.89 \\
            ROC & 0.92 & 0.92 & 0.93 & 0.94 \\
            \bottomrule
        \end{tabular*}
    \end{table}

    % MLP Cross-Performance
    \begin{table}[Ht!]
        \caption{Performance of Multilayer Perceptron model trained on dataset 1}\label{tab:performance_of_mlp_model_trained_on_dataset_1}
        \begin{tabular*}{\tblwidth}{@{}LLLLL@{}}
            \toprule
            Metric & Dataset 1 & Dataset 2 & Dataset 3 & Dataset 4 \\
            \midrule
            Accuracy & 0.97 & 0.96 & 0.96 & 0.96 \\
            F1 & 0.92 & 0.89 & 0.89 & 0.89 \\
            Precision & 0.97 & 0.95 & 0.94 & 0.95 \\
            Recall & 0.88 & 0.85 & 0.85 & 0.85 \\
            ROC & 0.93 & 0.92 & 0.92 & 0.92 \\
            \bottomrule
        \end{tabular*}
    \end{table}

    \begin{table}[Ht!]
        \caption{Performance of Multilayer Perceptron model trained on dataset 2}\label{tab:performance_of_mlp_model_trained_on_dataset_2}
        \begin{tabular*}{\tblwidth}{@{}LLLLL@{}}
            \toprule
            Metric & Dataset 1 & Dataset 2 & Dataset 3 & Dataset 4 \\
            \midrule
            Accuracy & 0.96 & 0.97 & 0.96 & 0.96 \\
            F1 & 0.90 & 0.91 & 0.90 & 0.90 \\
            Precision & 0.95 & 0.97 & 0.94 & 0.95 \\
            Recall & 0.85 & 0.87 & 0.85 & 0.85 \\
            ROC & 0.92 & 0.93 & 0.92 & 0.92 \\
            \bottomrule
        \end{tabular*}
    \end{table}

    \begin{table}[Ht!]
        \caption{Performance of Multilayer Perceptron model trained on dataset 3}\label{tab:performance_of_mlp_model_trained_on_dataset_3}
        \begin{tabular*}{\tblwidth}{@{}LLLLL@{}}
            \toprule
            Metric & Dataset 1 & Dataset 2 & Dataset 3 & Dataset 4 \\
            \midrule
            Accuracy & 0.96 & 0.96 & 0.96 & 0.96 \\
            F1 & 0.89 & 0.89 & 0.90 & 0.88 \\
            Precision & 0.89 & 0.89 & 0.90 & 0.89 \\
            Recall & 0.88 & 0.88 & 0.91 & 0.88 \\
            ROC & 0.93 & 0.93 & 0.94 & 0.93 \\
            \bottomrule
        \end{tabular*}
    \end{table}

    \begin{table}[Ht!]
        \caption{Performance of Multilayer Perceptron model trained on dataset 4}\label{tab:performance_of_mlp_model_trained_on_dataset_4}
        \begin{tabular*}{\tblwidth}{@{}LLLLL@{}}
            \toprule
            Metric & Dataset 1 & Dataset 2 & Dataset 3 & Dataset 4 \\
            \midrule
            Accuracy & 0.95 & 0.95 & 0.95 & 0.95 \\
            F1 & 0.85 & 0.85 & 0.85 & 0.86 \\
            Precision & 0.97 & 0.97 & 0.97 & 0.98 \\
            Recall & 0.75 & 0.75 & 0.75 & 0.76 \\
            ROC & 0.87 & 0.87 & 0.87 & 0.88 \\
            \bottomrule
        \end{tabular*}
    \end{table}

    % RF Cross-Performance
    \begin{table}[Ht!]
        \caption{Performance of Random Forest model trained on dataset 1}\label{tab:performance_of_rf_model_trained_on_dataset_1}
        \begin{tabular*}{\tblwidth}{@{}LLLLL@{}}
            \toprule
            Metric & Dataset 1 & Dataset 2 & Dataset 3 & Dataset 4 \\
            \midrule
            Accuracy & 0.99 & 0.97 & 0.97 & 0.97 \\
            F1 & 0.98 & 0.91 & 0.91 & 0.91 \\
            Precision & 0.99 & 0.98 & 0.97 & 0.98 \\
            Recall & 0.96 & 0.85 & 0.85 & 0.85 \\
            ROC & 0.98 & 0.92 & 0.92 & 0.92 \\
            \bottomrule
        \end{tabular*}
    \end{table}

    \begin{table}[Ht!]
        \caption{Performance of Random Forest model trained on dataset 2}\label{tab:performance_of_rf_model_trained_on_dataset_2}
        \begin{tabular*}{\tblwidth}{@{}LLLLL@{}}
            \toprule
            Metric & Dataset 1 & Dataset 2 & Dataset 3 & Dataset 4 \\
            \midrule
            Accuracy & 0.96 & 0.99 & 0.97 & 0.97 \\
            F1 & 0.90 & 0.97 & 0.91 & 0.90 \\
            Precision & 0.98 & 0.99 & 0.97 & 0.98 \\
            Recall & 0.83 & 0.96 & 0.85 & 0.84 \\
            ROC & 0.91 & 0.98 & 0.92 & 0.91 \\
            \bottomrule
        \end{tabular*}
    \end{table}

    \begin{table}[Ht!]
        \caption{Performance of Random Forest model trained on dataset 3}\label{tab:performance_of_rf_model_trained_on_dataset_3}
        \begin{tabular*}{\tblwidth}{@{}LLLLL@{}}
            \toprule
            Metric & Dataset 1 & Dataset 2 & Dataset 3 & Dataset 4 \\
            \midrule
            Accuracy & 0.96 & 0.97 & 0.99 & 0.97 \\
            F1 & 0.90 & 0.90 & 0.97 & 0.90 \\
            Precision & 0.98 & 0.98 & 0.99 & 0.98 \\
            Recall & 0.83 & 0.84 & 0.96 & 0.83 \\
            ROC & 0.91 & 0.91 & 0.98 & 0.91 \\
            \bottomrule
        \end{tabular*}
    \end{table}

    \begin{table}[Ht!]
        \caption{Performance of Random Forest model trained on dataset 4}\label{tab:performance_of_rf_model_trained_on_dataset_4}
        \begin{tabular*}{\tblwidth}{@{}LLLLL@{}}
            \toprule
            Metric & Dataset 1 & Dataset 2 & Dataset 3 & Dataset 4 \\
            \midrule
            Accuracy & 0.96 & 0.97 & 0.97 & 0.99 \\
            F1 & 0.90 & 0.91 & 0.90 & 0.97 \\
            Precision & 0.98 & 0.98 & 0.97 & 0.99 \\
            Recall & 0.84 & 0.84 & 0.85 & 0.95 \\
            ROC & 0.91 & 0.92 & 0.92 & 0.97 \\
            \bottomrule
        \end{tabular*}
    \end{table}

    % SVM Cross-Performance
    \begin{table}[Ht!]
        \caption{Performance of Support Vector Machine model trained on dataset 1}\label{tab:performance_of_svm_model_trained_on_dataset_1}
        \begin{tabular*}{\tblwidth}{@{}LLLLL@{}}
            \toprule
            Metric & Dataset 1 & Dataset 2 & Dataset 3 & Dataset 4 \\
            \midrule
            Accuracy & 0.97 & 0.97 & 0.97 & 0.97 \\
            F1 & 0.93 & 0.92 & 0.91 & 0.91 \\
            Precision & 0.98 & 0.97 & 0.96 & 0.96 \\
            Recall & 0.89 & 0.87 & 0.87 & 0.87 \\
            ROC & 0.94 & 0.93 & 0.93 & 0.93 \\
            \bottomrule
        \end{tabular*}
    \end{table}

    \begin{table}[Ht!]
        \caption{Performance of Support Vector Machine model trained on dataset 2}\label{tab:performance_of_svm_model_trained_on_dataset_2}
        \begin{tabular*}{\tblwidth}{@{}LLLLL@{}}
            \toprule
            Metric & Dataset 1 & Dataset 2 & Dataset 3 & Dataset 4 \\
            \midrule
            Accuracy & 0.97 & 0.98 & 0.97 & 0.97 \\
            F1 & 0.91 & 0.94 & 0.91 & 0.91 \\
            Precision & 0.97 & 0.98 & 0.96 & 0.96 \\
            Recall & 0.86 & 0.89 & 0.86 & 0.86 \\
            ROC & 0.92 & 0.94 & 0.93 & 0.92 \\
            \bottomrule
        \end{tabular*}
    \end{table}

    \begin{table}[Ht!]
        \caption{Performance of Support Vector Machine model trained on dataset 3}\label{tab:performance_of_svm_model_trained_on_dataset_3}
        \begin{tabular*}{\tblwidth}{@{}LLLLL@{}}
            \toprule
            Metric & Dataset 1 & Dataset 2 & Dataset 3 & Dataset 4 \\
            \midrule
            Accuracy & 0.97 & 0.97 & 0.98 & 0.97 \\
            F1 & 0.91 & 0.92 & 0.94 & 0.91 \\
            Precision & 0.97 & 0.97 & 0.98 & 0.97 \\
            Recall & 0.85 & 0.87 & 0.89 & 0.87 \\
            ROC & 0.92 & 0.93 & 0.94 & 0.93 \\
            \bottomrule
        \end{tabular*}
    \end{table}

    \begin{table}[Ht!]
        \caption{Performance of Support Vector Machine model trained on dataset 4}\label{tab:performance_of_svm_model_trained_on_dataset_4}
        \begin{tabular*}{\tblwidth}{@{}LLLLL@{}}
            \toprule
            Metric & Dataset 1 & Dataset 2 & Dataset 3 & Dataset 4 \\
            \midrule
            Accuracy & 0.97 & 0.97 & 0.97 & 0.97 \\
            F1 & 0.91 & 0.91 & 0.91 & 0.93 \\
            Precision & 0.97 & 0.96 & 0.96 & 0.98 \\
            Recall & 0.85 & 0.86 & 0.86 & 0.88 \\
            ROC & 0.92 & 0.93 & 0.93 & 0.94 \\
            \bottomrule
        \end{tabular*}
    \end{table}
    % \FloatBarrier
    \clearpage
}


% Numbered list
% Use the style of numbering in square brackets.
% If nothing is used, default style will be taken.
%\begin{enumerate}[a)]
%\item 
%\item 
%\item 
%\end{enumerate}  

% Unnumbered list
%\begin{itemize}
%\item 
%\item 
%\item 
%\end{itemize}  

% Description list
%\begin{description}
%\item[]
%\item[] 
%\item[] 
%\end{description}  

% Figure
% \begin{figure}[<options>]
% 	\centering
% 		\includegraphics[<options>]{}
% 	  \caption{}\label{fig1}
% \end{figure}


% \begin{table}[<options>]
% \caption{}\label{tbl1}
% \begin{tabular*}{\tblwidth}{@{}LL@{}}
% \toprule
%   &  \\ % Table header row
% \midrule
%  & \\
%  & \\
%  & \\
%  & \\
% \bottomrule
% \end{tabular*}
% \end{table}

% Uncomment and use as the case may be
%\begin{theorem} 
%\end{theorem}

% Uncomment and use as the case may be
%\begin{lemma} 
%\end{lemma}

%% The Appendices part is started with the command \appendix;
%% appendix sections are then done as normal sections
%% \appendix

% \section{}\label{}

\nocite{*}

% To print the credit authorship contribution details
\printcredits
\clearpage

%% Loading bibliography style file
%\bibliographystyle{model1-num-names}
\raggedright
\bibliographystyle{cas-model2-names}

% Loading bibliography database
\bibliography{cas-refs.bib}

% Biography
\bio{}
% Here goes the biography details.
\endbio

% \bio{pic1}
% Here goes the biography details.
\endbio

\end{document}

