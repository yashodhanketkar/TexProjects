\section{Literature Review}\label{sec:literature_review}

Supervised learning algorithms are widely used in various fields. \cite*{ref_paper_14} conducted a study on the application of machine learning in various industries. \citeauthor{ref_paper_14} suggests that various industries already use supervised learning algorithms to solve various problems. \cite*{ref_paper_6} used machine learning in Civil Structural Health Monitoring (SHM). \citeauthor{ref_paper_6} used supervised learning methods due to the abundance of well-labeled data. \citeauthor{ref_paper_6} concluded that machine learning showed satisfactory results.

Supervised learning algorithms are very efficient in conventional problems. \cite*{ref_paper_9} used machine learning to detect the DDOS attack. \citeauthor{ref_paper_9} used five main security factors as features. \citeauthor{ref_paper_9} conclude that the random forest algorithm was able to detect intrusion with high accuracy.

\cite*{ref_paper_2} conducted a study on the automated selection of SVM algorithms. The goal of the study was to select the most optimal SVM model for a given task. \citeauthor{ref_paper_2} suggested that hyperparameters and used data have an extreme impact on prediction time.

\cite*{ref_paper_38} used machine learning to detect arrhythmia. \citeauthor{ref_paper_38} concluded that SVM achieved up to 91.2\% accuracy. \cite*{ref_paper_16} also used machine learning to detect arrhythmia. The neural network provided good results.

Recently machine learning was used in the diagnosis of COVID-19 disease and study related to its structure. \cite*{ref_paper_20} surveyed the research conducted on the machine learning approach to diagnose the COVID-19. \citeauthor{ref_paper_20} suggested that the supervised learning algorithms showed promising results in diagnosis. \citeauthor{ref_paper_20} suggested that various types of datasets were used in these studies.

Machine learning is used in the diagnosis process. \cite*{ref_paper_34} machine learning is used for the detection of Parkinson's disease. The machine learning algorithm achieved 100\% accuracy with a 95-99\% confidence level. The RF and SVM showed better performance compared to other algorithms. In their paper \cite*{ref_paper_15} suggest that the inclusion of machine learning in diagnosis can lead to better results and higher accuracy. \citeauthor{ref_paper_15} mentions that machine learning can handle a high volume of data.

\cite*{ref_paper_27} reviewed 200 studies published about the use of machine learning in Parkinson's disease diagnosis. \citeauthor{ref_paper_27} found out that the use of machine learning improved clinical decisions. In thier study \cite*{ref_paper_30} used machine learning for the detection of Parkinson's disease. The SVM model outperformed other models. \citeauthor{ref_paper_30} concludes that machine learning provides a better detection method.

Arrhythmias are a symptom of cardiological disorders. \cite*{ref_paper_28} used a supervised learning method for the detection of arrhythmia. The machine learning system provided satisfactory results. Results showed high accuracy and sensitivity. \citeauthor{ref_paper_28} remark on the need for good feature selection and selection guidelines for better-performing algorithms.

The machine learning system needs to be robust, extremely accurate, and easy to use. In the field of healthcare, according to \cite*{ref_paper_4} these are essential requirements. In this study, \citeauthor{ref_paper_4} used machine learning to classify arrhythmia. \citeauthor{ref_paper_4} remark that early detection of arrhythmia is critical for better treatment.

\cite*{ref_paper_32} conducted a study on the automatic selection of machine learning algorithms and their hyperparameters. \citeauthor{ref_paper_32} showed limitations in the biomedical industry with the help of machine learning systems. \cite*{ref_paper_37} studied the security and privacy aspect of machine learning solutions. \citeauthor{ref_paper_37} suggest that while machine learning has great potential in the healthcare system, more research about security and privacy aspects is necessary.

Machine learning utilizes various approaches for the same problem. In thier study \cite*{ref_paper_8} used two approaches to solve a diagnosis problem. In a direct approach, data was fed to a machine learning model. In an indirect approach, data was equalized before the machine learning process. The indirect method showed better results compared to the direct method. While the experiment was successful, \citeauthor{ref_paper_8} suggests that machine learning is still unstable for the medical field.

The supervised learning methods are very effective in the medical field. \cite*{ref_paper_11} suggest that ensembled supervised learning systems can further improve effectiveness. \citeauthor{ref_paper_11} further remark that machine learning will reduce the number of errors caused by humans.

In the review about the use of machine learning in the medical field, \cite*{ref_paper_33} suggested that big companies are already using machine learning for various tasks. \citeauthor{ref_paper_33} suggested that a machine learning system needs to be handled by non-technical people and it should support various ranges and types of data.

The automated system will allow non-technical people to use machine learning. \cite*{ref_paper_3} studied the automatic model selection for optimal SVM kernels. \citeauthor{ref_paper_3} used different automation approaches for optimal hyperparameter calculations. \citeauthor{ref_paper_3} state that the system can calculate up to two parameters without human interaction.

The automatic selection process can be extremely beneficial in dynamic environments. The excavation of soil or tunneling is one such environment. To predict the displacement induced by excavation, \cite*{ref_paper_1} used machine learning. \citeauthor{ref_paper_1} used properties of soli and imputed them as features. \citeauthor{ref_paper_1} concluded that the unsupervised machine learning algorithm GA-MLP showed good potential, while AutoML is found to be the most optimal algorithm in these dynamic conditions. \cite*{ref_paper_13} also suggested that machine learning can be used in dynamic environments successfully.

There are multiple ways to select the best-suited model, the important part of such a system is the selection system. The system uses various approaches for selection, ranking is one of those approaches. In thier study \cite*{ref_paper_23} suggest that the meta ranking is better than the baseline ranking system. \citeauthor{ref_paper_23} used a multi-criteria ranking system for the selection of ideal classifiers. \citeauthor{ref_paper_23} suggest that this study laid some groundwork for future automated machine learning selection systems.
