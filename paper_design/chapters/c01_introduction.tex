\section{Introduction}\label{sec:introduciton}

In recent years, machine learning has become very popular. Earlier, machine learning was limited to the research field. But quite recently, it has been used in various fields. This is partially due to the increasing availability of computers and growth in computing power.

The researchers are using machine learning in various fields. One of such fields is solar energy production. In this field, the prediction of solar radiation available per day can be very significant. \cite*{ref_paper_7} employed a few machine learning algorithms for the prediction of the amount of solar radiation received in a day. \citeauthor{ref_paper_7} used natural properties such as weather, time, etc as features for the machine learning algorithms. \citeauthor{ref_paper_7} concluded that supervised learning models gave satisfactory results in predicting the amount of solar radiation. \citeauthor{ref_paper_7} concluded that the ANN model showed great potential for such specialized tasks.

In the field of chemistry, there is a wide range of factors that need to be considered. \cite*{ref_paper_10} used machine learning in the field of chemistry. \citeauthor{ref_paper_10} suggest that the models performed better than conventional statistical methods. \citeauthor{ref_paper_10} concluded that more research is necessary before implementation.

Machine learning can be applied in network security for intrusion detection. \cite*{ref_paper_21} state that the supervised learning methods perform satisfactorily for intrusion detection. \citeauthor{ref_paper_21} also remark that the supervised learning methods are limited to conventional problems.

\cite*{ref_paper_36} in thier study used machine learning models to solve real-life problems. \citeauthor{ref_paper_36} concluded that machine learning plays an important role in good decision-making. \citeauthor{ref_paper_36} also remark that machine learning can be used in various fields.

\cite*{ref_paper_24} note a lack of information and meaningful research in the medical field. \citeauthor{ref_paper_24} suggest that more support, subject-specific scope, and accuracy are important factors for the higher impact of machine learning in the medical field. \citeauthor{ref_paper_24} also suggest that the ideal system should be able to take multiple data types for training.

\cite*{ref_paper_29} used machine learning with a combination of IoT for smart cities. \citeauthor{ref_paper_29} concludes that machine learning showed promising results for smart cities. \citeauthor{ref_paper_29} remark that machine learning was able to handle the high volume of data generated by sensors. \cite*{ref_paper_12} in thier paper suggests that machine learning can handle downstream tasks efficiently.

This suggests that even with the immense popularity and accessibility of machine learning, it is still underutilized by the general population. This can be attributed to lack of knowledge and skills. To solve this problem we are proposing a system with minimum user interaction. The proposed system can be operated by users without prior knowledge of machine learning. This system allows them to train their machine learning algorithms.

The proposed system can be deployed on a local network. The system can feed data manually or automatically according to user needs and policies. The local deployment also limits external access and reduces the influence of external factors on data.
