\thispagestyle{fancy}
\chapter{Introduction} \label{ch:introduction}
\section*{\centering Chapter \thechapter}
\section*{\centering Introduction}

\section{Introduction}

In recent years COVID-19 has become extremely widespread. It attacks the human respiratory
system and leads to complicated respiratory problems. This disease can be very severe depending
on a person's health and presence of comorbidities. Cardiovascular comorbidities are very
common in severe cases of COVID-19 disease. Cardiovascular comorbidities are also harder to
detect without proper equipment. One detection method is to check the presence of arrhythmia in
patients. Arrhythmia is improper beating of the heart, whether irregular, too fast or too slow.
This can be detected by reading ECG signals.  As COVID-19 has put a lot of stress on the
medical staff so detection is taking more time than usual.   Increased access to the Internet
has led various industries to implement machine learning and AI for  service. This leads to the
growth of research in the field of machine learning and will affect machine learning in various
fields. The medical and healthcare industries are one of them. Machine learning is  used to
quickly detect a patient's illness and classify viruses and other microbes. Machine learning
algorithms have already proven to be very effective in medical applications.

The reading of these ECG signals and detection can be done with help of the machine learning
system. The supervised learning algorithms can be used for detection of these symptoms in a
much faster and more efficient way. The supervised algorithms are already proven faster in such
precise classification tasks than unsupervised algorithms. Once trained this algorithm can be
used for future predictions too.

There are multiple supervised algorithms available, so we can choose algorithms that suit our
needs. In case of the general populace this step can be automated. The preset of a few
algorithms can be provided to the system, then the machine can train and select the best model
selected for the provided dataset. This will allow medical staff to focus on patient care and
develop solutions to medical problems.


\section{Motivation} \label{sec:motivation}

COVID-19 disease is a global pandemic which caused more than 250 million infections world-wide
and 5 million deaths. Followed by respiratory disorders, the cardiovascular diseases and
cardiovascular complications lead to the most severe cases and are one of most common causes of
deaths in COVID-19 patients. Due to the nature of cardiovascular disease they often go
undetected and untreated. This further leads to risks of severe COVID-19 in those individuals.

Earlier diagnosis of cardiovascular disease can lead to lower severity of disease and reduce
the mortality rate. Arrhythmia is one of the symptoms of cardiovascular disease and
cardiovascular complications. ECG signals are used in detection of cardiovascular disease. But
this process is very time consuming and can lead to a lot of human errors. Due to strain on the
healthcare system this has become an incredibly difficult task. The implementation of machine
learning can lead to faster detection of arrhythmia. This will help in early diagnosis and
treatment of cardiovascular problems in COVID-19 patients and will lead to good outcomes.


\section{Beneficiaries} \label{sec:beneficiaries}

This system is helpful for medical personnels handling covid-19 cases. This system will also
help other patients that suffer from cardiovascular problems or other problems.

The system can be used for a wide range of applications, such as detection of epilepsy from
frequency charts obtained during testing, also detection of other disorders. This system will
reduce workload from medical staff and allow for better understanding of problems present in
patients' cases.

This system can be trained once for generalized predictions, or used for unique purposes
depending on users needs. The system does not need machine learning experts to operate, it can
be trained and used by end users directly, hence providing an additional layer of security.


\section{Problem Statement} \label{sec:problem_statement}
In current COVID-19 pandemic life threatening complications can be avoided with earlier
diagnosis. One of such complications is arrhythmia which can be detected by reading ECG signals.
This task is carried out by medical personnel. This repetitive and mundane task puts strain on
medical staff.

Supervised learning models can be used for such tasks. These models have their own set
advantages and disadvantages. Hence automating selection of the best model for required tasks
will allow medical personnel to customise applications based on need. This will reduce strain
from medical staff and allow them to focus on patient care.

\section{Scope of the project} \label{sec:scope_of_the_project}

To make the system more efficient, data used for training is provided in csv format. And
obtained results are stored in json format. The system handles imputed data with the help of
the pandas library. The dataset provided needs to be specific format regardless of length and
parameters present in the dataset. The datasets used in this projects are available on kaggle
and are converted in required format, but if needed the raw datasets obtained directly from
machines can be used for operational purposes.

\section{Organization of Report} \label{sec:organization_of_report}

The report is structure into seven chapters as follows:

\textbf{\textit{Chapter \ref{ch:introduction} - \nameref{ch:introduction}:}}
This chapter gives an overview of the report. It gives a general overview of the current
situation of COVID-19 pandemic and the stress it puts on the medical system. It also describes
the proposed solution to tackle this problem.

\textbf{\textit{Chapter \ref{ch:literature_survey} - \nameref{ch:literature_survey}:}}
This chapter provides review of relevant literature of supervised learning systems and
automation approaches for the model training.

\textbf{\textit{Chapter \ref{ch:system_architecture} - \nameref{ch:system_architecture}:}}
This chapter describes the architecture of the system. It discusses the models used in the
system, the automated training and selection of models.

\textbf{\textit{Chapter \ref{ch:system_interface} - \nameref{ch:system_interface}:}}
This chapter describes the user interface of the system. This chapter will discuss the pages
accessible by the user and the information about the development server.

\textbf{\textit{Chapter \ref{ch:result_and_analysis} - \nameref{ch:result_and_analysis}:}}
This chapter will describe the performance metrics used by the system for the selection
process. The chapter also gives details about the dataset used for training and testing. It
also provides recommended system specification required by the application.

\textbf{\textit{Chapter \ref{ch:applications} - \nameref{ch:applications}:}}
This chapter gives various applications of the developed system.

\textbf{\textit{Chapter \ref{ch:conclusion_and_future_scope} - \nameref{ch:conclusion_and_future_scope}:}}
This chapter concludes the report by providing a brief summary of the project. This chapter
also describes the future scope of the system.
