\section*{\centering Abstract}

COVID-19 disease has become a pandemic over the past year. This disease is highly contagious and
spread throughout many countries. This disease with combination of comorbidities can lead to serious
consequences for its victims. Cardiovascular comorbidities are leading to serious illness and death
in COVID-19 infection. These cardiovascular comorbidities can also manifest after COVID-19 infection.
These Cardiovascular problems can be detected using supervised learning methods, faster and more
efficiently than traditional methods. Method used in this report uses supervised learning methods for
detection of these problems. The method automatically trains and selects the best model for data
provided and stores this data for future predictions. This will allow medical staff to focus on
patient care and develop solutions faster. The implemented system uses  K-Nearest Neighbors (KNN),
Decision Tree (DT), Random Forest Algorithm (RF), Multi-layer Perceptron (MLP) and Support Vector
Machine (SVM). The proposed system measures performance of this algorithm with respect to accuracy,
F1, recall, precision, ROC and prediction time for each item.