\thispagestyle{fancy}
\chapter{Literature Survey} \label{ch:literature_survey}
\section*{\centering Chapter \thechapter}
\section*{\centering Literature Survey}
% \nocite{*}

\section{Introduction}\label{sec:introduction_literature_review}

COVID-19 has become a global pandemic in the past few years. It targeted the human respiratory system and caused serious infection in the patients. This infection affected patients quite severely in the case of the present comorbidities. This resulted in a fatal infection in some cases. These comorbidities were mainly diabetes, cardiovascular disease, prior respiratory system damage, etc. Cardiovascular diseases, also known as CVDs, are caused by disorders in the cardiovascular system, including the heart and other blood-circulating organs. Major Cardiovascular diseases are life-threatening, while minor Cardiovascular diseases can lead to heart attacks or strokes. The presence of this disease can lead to severe COVID-19 infection.

Cardiovascular disease is generally accompanied by various symptoms, but arrhythmia is the most common symptom among them. \cite{ref_paper_covid_1} suggest that arrhythmia is one of the most common symptoms in COVID-19 patients. An arrhythmia is an irregular beating of the heart. This can be easily detected by the ECG report from the patients. The ECG graphs are usually recorded at a 128 Hz frequency over a few minutes. These graphs are studied by the medical staff for the detection of arrhythmia. If any irregularity is present, the patient is said to have arrhythmia. This process can be easily automated by the machine learning system.

\citeauthor{ref_paper_covid_1} also suggests that arrhythmia was present in about 7\% of total COVID-19 cases in Wuhan and 14.8\% of patients with poor outcomes had arrhythmia. The authors of \cite{ref_paper_covid_2,ref_paper_covid_3} carried out seperate surveys of 17 studies consists of 5815 patients are concluded similar findings that up to 9.3\% of COVID-19 confirmed patients had been detected with arrhythmia. The surveys \cite{ref_paper_covid_4, ref_paper_covid_5} showed that up to 94.4\% of patients with arrhythmia resulted in fatal infection and 95.8\% of patients with severe infections had arrhythmia. In the \cite{ref_paper_covid_6}, author suggests that only 8\% of patients with arrhythmia had prior cardiovascular disease, while in 56\% of the cases, arrhythmia symptoms were new onset after contracting COVID-19 disease.

\section{Machine Learning}\label{sec:machine_learning}

Machine learning is becoming a widespread technology. Machine learning is a technology where the computers learn to make decisions based on provided data. Machine learning uses statistical modeling and advanced mathematics to build decision making models.

The industries are making large amounts of data every day. Making sense of this data will give an advantage over the competitors. Machine learning technology is capable of this feat. Hence, many industries are already using machine learning. The medical industry is one of them. The medical industry uses machine learning for the research and detection of various diseases. Well trained machine learning models are extremely accurate and fast. This allows faster diagnosis, which leads to better patient care. Supervised machine learning and unsupervised learning are two main types of machine learning systems.

\subsection{Supervised Machine Learning}\label{subsec:supervised_machine_learning}

Supervised machine learning is a subset of machine learning technology. The supervised learning system uses the labeled data to derive meaningful results. The computer learns the patterns from the data and uses labels to increase the accuracy of the models. The supervised models become more efficient with large amounts of data. Supervised learning models are generally used for classification and regression problems.

\section{Review of related literature}\label{sec:review_of_related_literature}

\subsection{Prediction of daily global solar radiation using different machine learning algorithms: Evaluation and comparison}
\subsubsection{\citeauthor*{ref_paper_7} \citeyearpar{ref_paper_7}}

The authors used deep learning, SVM and KNN for calculation of daily global solar radiation. The radiation is directly proportional to the amount of solar power generated, hence correct assessment gives edge over the competition. The authors used various natural conditions as variables to correctly predict the amount of radiation per day. Authors suggested that neural networks are extremely accurate for their use cases.

\subsection{A fast machine learning model for ECG-based heartbeat classification and arrhythmia detection}
\subsubsection{\citeauthor*{ref_paper_28} \citeyearpar{ref_paper_28}}

The authors used machine learning for the detection of the arrhythmia from ECG signals. The author used a single lead recording over a long period of time. The authors achieved up to 92.7\% sensitivity and 86\% accuracy in the first test, while 95.7\% sensitivity and 75\% accuracy in the second test. The authors utilized a GPU for the calculations to reduce the amount of training and prediction time. The authors suggest that good feature selection and guidelines are necessary for efficient machine learning models.

\subsection{Automatic model selection for the optimization of SVM kernels}
\subsubsection{\citeauthor*{ref_paper_3} \citeyearpar{ref_paper_3}}

The authors used novel criteria for model selection against the custom SVM classifier. The system needs prior knowledge of machine learning technology. The system was able to handle the data up to two parameters independently. The system was unable to handle multi class data, hence it should be partitioned appropriately. The automation system reduced the probability of errors.

\subsection{Ranking learning algorithms: Using IBL and meta-learning on accuracy and time results}
\subsubsection{\citeauthor*{ref_paper_23} \citeyearpar{ref_paper_23}}

The authors used a meta ranking approach for the selection of machine learning algorithms instead of the baseline approach. The models are ranked according to the features of the data. The methodology used for the ranking is based on the success rate of the model and time parameter. The author suggested that the study will be beneficial to develop future ranking methods.

\subsection{A survey on the explainability of supervised machine learning}
\subsubsection{\citeauthor*{ref_paper_14} \citeyearpar{ref_paper_14}}

The authors conducted a survey of the machine learning system in finance, healthcare and various other industries. This survey provided the overview of supervised learning algorithms. The author suggests that the uncertainty about AI and machine learning is holding industrialization back. Also extremely complicated models are required for the complex industrial tasks making it harder to adopt. The survey suggested that the machine learning system was widely used during COVID-19 pandemic. Changing the approach towards the solution will increase the adaptation of machine learning. Also setting up clear ethical clauses in machine learning is necessary for wide use of this technology.

\subsection{Machine learning in medicine}
\subsubsection{\citeauthor*{ref_paper_24} \citeyearpar{ref_paper_24}}

The author commented on the lack of meaningful research of machine learning in the medical field. The authors studied both supervised and unsupervised machine learning algorithms. The author concluded that the supervised learning methods are beneficial in risk assessment and detection of known diseases, while the unsupervised methods are beneficial in the same task in case of the novel diseases. The author suggested that the large amount of data is necessary for accurate predictions. The subject specific methods as well as wider support is necessary for wide scale use of machine learning in the medical field.

\subsection{Machine learning in detection and classification of leukemia using smear blood images: a systematic review}
\subsubsection{\citeauthor*{ref_paper_15} \citeyearpar{ref_paper_15}}

The authors reviewed studies conducted to detect leukemia from blood smears with machine learning. The authors suggest that machine learning eased the load on the personnel. The machine learning systems achieved up to 97\% accuracy in leukemia detection in a case. Machine learning also produced more than 74\% accuracy in other cases.

\subsection{IoT for smart cities: Machine learning approaches in smart healthcare - A review}
\subsubsection{\citeauthor*{ref_paper_29} \citeyearpar{ref_paper_29}}

The authors used machine learning systems with IoT systems for smart cities. The authors suggested that sensors provide a large amount of data over time for the machine learning system. Authors suggest that machine learning will be extremely beneficial in the development of smart cities, advanced healthcare systems, and various other fields.

\subsection{Battling COVID-19 using machine learning: A review}
\subsubsection{\citeauthor*{ref_paper_20} \citeyearpar{ref_paper_20}}

The authors used a machine learning system for detection of viral disease. The machine learning system showed promising results. The authors are planning to do more research with various datasets to evaluate the efficiency and accuracy of machine learning.

\subsection{The role of machine learning algorithms for diagnosing diseases}
\subsubsection{\citeauthor*{ref_paper_11} \citeyearpar{ref_paper_11}}

The authors studied various machine learning techniques used for the diagnosis of diseases. The author suggests that machine learning reduces the number of human errors while making the system useful for daily life. The author suggests that KNN, SVM, DT, and RF models showed good results. The author also suggested that ensembling methods would produce even better results.

\subsection{Classification of pachychoroid disease on ultrawide-field indocyanine green angiography using auto-machine learning platform}
\subsubsection{\citeauthor*{ref_paper_8} \citeyearpar{ref_paper_8}}

The authors employed an auto machine learning platform to classify the pachychoroid disease. The authors used direct and indirect approaches for comparisons. The authors found that the indirect approach of equalizing the data produced better results. The author was satisfied with the auto machine learning algorithms, but suggested that the technology is still unstable for proper medical uses.

\subsection{Automatic model selection for support vector machines}
\subsubsection{\citeauthor*{ref_paper_2} \citeyearpar{ref_paper_2}}

The authors proposed an automation system to generate the most-suited SVM models. Theory behind the research was the approximation of loo stopping rate was reducing the efficiency of the SVM models. The author suggested that better loo stopping will reduce the training and testing time. The authors used stalag collection after scaling appropriately for the model generation. The produced model is suitable for data upto 1000 variables. The system is currently limited to SVM models and RBF kernels.

\subsection{Deep transfer learning for industrial automation: a review and discussion of new techniques for data-driven machine learning}
\subsubsection{\citeauthor*{ref_paper_13} \citeyearpar{ref_paper_13}}

The authors used transfer learning in the industrial tasks for industrial automation systems. The authors suggested that transfer learning and continuous learning technologies are important for rapid industrialization of machine learning technologies. These will be obtained by reducing the gap between research and practical uses of these technologies. Author also suggests that machine learning performs better in dynamic environments of industries.

\subsection{Benchmarking of machine learning for anomaly based intrusion detection systems in the CICIDS2017 dataset}
\subsubsection{\citeauthor*{ref_paper_21} \citeyearpar{ref_paper_21}}

The authors used machine learning for the detection of anomalies. The DT, KNN, SVM and ANN algorithms were used for the tasks. The author also used some unsupervised learning algorithms, but supervised learning algorithms provided better results. The DT and KNN models provided bet results. The author concluded that supervised machine learning is still limited in case of complex and novel problems, but otherwise produced better results than unsupervised learning algorithms.

\subsection{DDOS detection using machine learning technique}
\subsubsection{\citeauthor*{ref_paper_9} \citeyearpar{ref_paper_9}}

The authors used random forest methods for detection of DDOS attacks. Five properties of the DDOS attacks are used as the variable in machine learning algorithms. The authors suggested that the machine learning model random forest produced good results.

\subsection{Classification of arrhythmia using machine learning techniques}
\subsubsection{\citeauthor*{ref_paper_4} \citeyearpar{ref_paper_4}}

The authors used machine learning to automatically classify the cardiac arrhythmias in embedded systems. The models were focused on three factors: accuracy, predictions and ease-of-use of the system. The author suggested that the volume of data is indirectly proportional to the understanding of the data. The designed model produced satisfactory results, hence proving the usefulness of machine learning in the medical field.

\subsection{Auto machine learning-based modelling and prediction of excavation-induced tunnel displacement}
\subsubsection{\citeauthor*{ref_paper_1} \citeyearpar{ref_paper_1}}

Authors proposed an AutoML system for the prediction of excavation of displacement. Six generic models were provided with seven input points and two properties. The system was implemented with 10-fold cross-validation methods to increase the efficiency. The models produced efficient and precise predictions. The authors concluded that AutoML is optimal for these predictive tasks. Authors also suggested that GA-MLP also provided satisfactory results.

\subsection{Predicting reaction yields via supervised learning}
\subsubsection{\citeauthor*{ref_paper_10} \citeyearpar{ref_paper_10}}

The authors used a machine learning system in case with a large number of features. Authors suggested that a similar result to traditional statistical methods was achieved with machine learning algorithms. The author suggested that good data collection techniques are necessary for better results. The authors suggested that use of machine learning methods are successful in the chemical domain. The author also suggests that more development is necessary to accommodate the vast amount of features in the chemical field.

\section{Inference of Literature Review} \label{sec:inference _of_literature_review}

The following inference can be drawn from the basis of literature review

\vspace{-1.5em}

\begin{itemize}
  \item Most papers introduced here use supervised learning algorithms for the prediction. This technique needs a well labeled dataset for training and evaluation. Therefore, expertknowledge is needed during the training and evaluation process.
  \item Some papers concluded that it is possible to select suitable models with minimum human interactions automatically. The knowledge driven and data driven approaches are used for automated model suggestions.
  \item Most papers also showed that on-shelf models can be used for accurate and efficient use cases, but few authors used novel kernels and models to get greater accuracy and efficiency for predictions.
  \item The recent paper showed that there is great need for models that can be setup faster for required work. To solve this problem we can use limited data for training which is possible with supervised learning algorithms according to few studies.
  \item Finally it can be inferred that a very negligible amount of work has been done for the automated model selection for general prediction purposes. Moreover, a limited amount of study is done with the general population as end users. Hence, this system is focusing on training, evaluation and selection of models for users with limited knowledge of data science.
\end{itemize}
