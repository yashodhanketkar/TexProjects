\thispagestyle{fancy}
\chapter{Literature Survey} \label{ch:literature_survey}
\section*{\centering Chapter \thechapter}
\section*{\centering Literature Survey}
% \nocite{*}

\section{Introduction}\label{sec:introduction_literature_review}

COVID-19 has become a global pandemic in the past few years. It targeted the human respiratory system and caused serious infection in the patients. This infection affected patients quite severely in the case of the present comorbidities. This resulted in a fatal infection in some cases. These comorbidities were mainly diabetes, cardiovascular disease, prior respiratory system damage, etc. Cardiovascular diseases, also known as CVDs, are caused by disorders in the cardiovascular system, including the heart and other blood-circulating organs. Major Cardiovascular diseases are life-threatening, while minor Cardiovascular diseases can lead to heart attacks or strokes. The presence of this disease can lead to severe COVID-19 infection.

Cardiovascular disease is generally accompanied by various symptoms, but arrhythmia is the most common symptom among them. \cite{ref_paper_covid_1} suggest that arrhythmia is one of the most common symptoms in COVID-19 patients. An arrhythmia is an irregular beating of the heart. This can be easily detected by the ECG report from the patients. The ECG graphs are usually recorded at a 128 Hz frequency over a few minutes. These graphs are studied by the medical staff for the detection of arrhythmia. If any irregularity is present, the patient is said to have arrhythmia. This process can be easily automated by the machine learning system.

\citeauthor{ref_paper_covid_1} also suggests that arrhythmia was present in about 7\% of total COVID-19 cases in Wuhan and 14.8\% of patients with poor outcomes had arrhythmia. The authors of \cite{ref_paper_covid_2,ref_paper_covid_3} carried out seperate surveys of 17 studies consists of 5815 patients are concluded similar findings that up to 9.3\% of COVID-19 confirmed patients had been detected with arrhythmia. The surveys \cite{ref_paper_covid_4, ref_paper_covid_5} showed that up to 94.4\% of patients with arrhythmia resulted in fatal infection and 95.8\% of patients with severe infections had arrhythmia. In the \cite{ref_paper_covid_6}, author suggests that only 8\% of patients with arrhythmia had prior cardiovascular disease, while in 56\% of the cases, arrhythmia symptoms were new onset after contracting COVID-19 disease.

\newpage
\section{Machine Learning}\label{sec:machine_learning}

Machine learning is becoming a widespread technology. Machine learning is a technology where the computers learn to make decisions based on provided data. Machine learning uses statistical modeling and advanced mathematics to build decision making models.

The industries are making large amounts of data every day. Making sense of this data will give an advantage over the competitors. Machine learning technology is capable of this feat. Hence, many industries are already using machine learning. The medical industry is one of them. The medical industry uses machine learning for the research and detection of various diseases. Well trained machine learning models are extremely accurate and fast. This allows faster diagnosis, which leads to better patient care. Supervised machine learning and unsupervised learning are two main types of machine learning systems.

\subsection{Supervised Machine Learning}\label{subsec:supervised_machine_learning}

Supervised machine learning is a subset of machine learning technology. The supervised learning system uses the labeled data to derive meaningful results. The computer learns the patterns from the data and uses labels to increase the accuracy of the models. The supervised models become more efficient with large amounts of data. Supervised learning models are generally used for classification and regression problems.

Supervised machine learning algorithms are widely used in various fields. Various industries have already been using supervised machine learning systems to gain competitive advantages. The authors of \cite{ref_paper_14} conducted a study of the industry and found that the supervised learning algorithms are well integrated into industrial environments. In the construction industry, supervised algorithms are used due to the abundance of well-labeled data. The authors of \cite{ref_paper_6} used supervised machine learning for civil structural health monitoring. \citeauthor{ref_paper_6} suggested that machine learning provided satisfactory results in the tasks. In the field of IT, DDOS attacks can destabilize a whole network. Machine learning systems have already been used to detect such attacks. The authors of \cite{ref_paper_9} used machine learning to detect DDOS attacks using the five security factors as features. \citeauthor{ref_paper_9} used random forest algorithms for the detection purpose. This system was able to detect intrusion with extremely high accuracy. In this way, supervised learning systems prove to be a very efficient solution for conventional industrial tasks.

\subsection{Machine learning in Healthcare}\label{subsec:machine_learning_in_healthcare}

\cite{ref_paper_38} used machine learning to detect arrhythmia. \cite{ref_paper_38} concluded that SVM achieved up to 91.2\% accuracy. \cite{ref_paper_16} also used machine learning to detect arrhythmia. The neural network provided good results.

Recently machine learning was used in the diagnosis of COVID-19 disease and study related to its structure. \cite{ref_paper_20} surveyed the research conducted on the machine learning approach to diagnose the COVID-19. \cite{ref_paper_20} suggested that the supervised learning algorithms showed promising results in diagnosis. \cite{ref_paper_20} suggested that various types of datasets were used in these studies.

Machine learning is used in the diagnosis process. \cite{ref_paper_34} machine learning is used for the detection of Parkinson's disease. The machine learning algorithm achieved 100\% accuracy with a 95-99\% confidence level. The RF and SVM showed better performance compared to other algorithms. In their paper \cite{ref_paper_15} suggest that the inclusion of machine learning in diagnosis can lead to better results and higher accuracy. \cite{ref_paper_15} mentions that machine learning can handle a high volume of data.

\cite{ref_paper_27} reviewed 200 studies published about the use of machine learning in Parkinson's disease diagnosis. \cite{ref_paper_27} found out that the use of machine learning improved clinical decisions. In thier study \cite{ref_paper_30} used machine learning for the detection of Parkinson's disease. The SVM model outperformed other models. \cite{ref_paper_30} concludes that machine learning provides a better detection method.

Arrhythmias are a symptom of cardiological disorders. \cite{ref_paper_28} used a supervised learning method for the detection of arrhythmia. The machine learning system provided satisfactory results. Results showed high accuracy and sensitivity. \cite{ref_paper_28} remark on the need for good feature selection and selection guidelines for better-performing algorithms.

The machine learning system needs to be robust, extremely accurate, and easy to use. In the field of healthcare, according to \cite{ref_paper_4} these are essential requirements. In this study, \cite{ref_paper_4} used machine learning to classify arrhythmia. \cite{ref_paper_4} remark that early detection of arrhythmia is critical for better treatment.

\cite{ref_paper_40} used ECG signals of COVID-19 patients for patient monitoring. \cite{ref_paper_40} used LSTM, SVM, and MLP algorithms to monitor data. The authors suggest that machine learning with robotics can provide better results.
\cite{ref_paper_42} used a multi perceptron neural network for stroke predictions. The neural network showed high accuracy. \cite{ref_paper_42} were able to achieve up to 78\% accuracy. \cite{ref_paper_42} suggested that the model can produce better results with a larger training dataset. \cite{ref_paper_41} used artificial intelligence to detect heart disease. \cite{ref_paper_41} concluded that the algorithms achieved up to 83\% accuracy. \cite{ref_paper_41} also concluded that the system was able to comply with the HIPPA regulations.

The supervised learning methods are very effective in the medical field. \cite{ref_paper_11} suggest that ensembled supervised learning systems can further improve effectiveness. \cite{ref_paper_11} further remark that machine learning will reduce the number of errors caused by humans.

In the review about the use of machine learning in the medical field, \cite{ref_paper_33} suggested that big companies are already using machine learning for various tasks. \cite{ref_paper_33} suggested that a machine learning system needs to be handled by non-technical people and it should support various ranges and types of data.

\cite{ref_paper_32} conducted a study on the automatic selection of machine learning algorithms and their hyperparameters. \cite{ref_paper_32} showed limitations in the biomedical industry with the help of machine learning systems. \cite{ref_paper_37} studied the security and privacy aspect of machine learning solutions. \cite{ref_paper_37} suggest that while machine learning has great potential in the healthcare system, more research about security and privacy aspects is necessary.

Machine learning utilizes various approaches for the same problem. In thier study \cite{ref_paper_8} used two approaches to solve a diagnosis problem. In a direct approach, data was fed to a machine learning model. In an indirect approach, data was equalized before the machine learning process. The indirect method showed better results compared to the direct method. While the experiment was successful, \cite{ref_paper_8} suggests that machine learning is still unstable for the medical field.

\section{Automated model selection approaches} \label{sec:automated_model_selection_approaches}

\cite{ref_paper_2} conducted a study on the automated selection of SVM algorithms. The goal of the study was to select the most optimal SVM model for a given task. \cite{ref_paper_2} suggested that hyperparameters and used data have an extreme impact on prediction time.

The automated system will allow non-technical people to use machine learning. \cite{ref_paper_3} studied the automatic model selection for optimal SVM kernels. \cite{ref_paper_3} used different automation approaches for optimal hyperparameter calculations. \cite{ref_paper_3} state that the system can calculate up to two parameters without human interaction. \cite{ref_paper_39} state that the model parameters can be optimized with a genetic algorithm. \cite{ref_paper_39} successfully used a genetic algorithm to reduce uncertainty from the prediction results.

The automatic selection process can be extremely beneficial in dynamic environments. The excavation of soil or tunneling is one such environment. To predict the displacement induced by excavation, \cite{ref_paper_1} used machine learning. \cite{ref_paper_1} used properties of soli and imputed them as features. \cite{ref_paper_1} concluded that the unsupervised machine learning algorithm GA-MLP showed good potential, while AutoML is found to be the most optimal algorithm in these dynamic conditions. \cite{ref_paper_13} also suggested that machine learning can be used in dynamic environments successfully.

There are multiple ways to select the best-suited model, the important part of such a system is the selection system. The system uses various approaches for selection, ranking is one of those approaches. In thier study \cite{ref_paper_23} suggest that the meta ranking is better than the baseline ranking system. \cite{ref_paper_23} used a multi-criteria ranking system for the selection of ideal classifiers. \cite{ref_paper_23} suggest that this study laid some groundwork for future automated machine learning selection systems.

In the paper \cite{ref_paper_43}, the authors used 2 years of data collected from Glumo Lake. Expert knowledge is used for model training and selection. The mixed approach of data driven and knowledge driven modeling is used for successful application. While the authors of the paper, \cite{ref_paper_2}, used loo rate and stopping criteria for model selection. The author considered eight different problems and discovered that a larger loo rate is more desirable. The author also suggested that modeling difficulties can only be found with careful numerical calculations.

In the paper by \cite{ref_paper_44}, novel kernels are used to obtain explanations of the dataset. This approach leads to the discovery of unseen models with minimal human interaction. The authors of the paper, \cite{ref_paper_45}, used the glmulti package to build new models. These models are unique and flexible. The models are optimized automatically and provide a multi-model interface. With this approach, a large set of models can be considered for selection purposes at a rapid rate.

\section{Review of related literature}\label{sec:review_of_related_literature}

\subsection{Prediction of daily global solar radiation using different machine learning algorithms: Evaluation and comparison}
\subsubsection{\citeauthor*{ref_paper_7} \citeyearpar{ref_paper_7}}

The authors used deep learning, SVM and KNN for calculation of daily global solar radiation. The radiation is directly proportional to the amount of solar power generated, hence correct assessment gives edge over the competition. The authors used various natural conditions as variables to correctly predict the amount of radiation per day. Authors suggested that neural networks are extremely accurate for their use cases.

\subsection{A fast machine learning model for ECG-based heartbeat classification and arrhythmia detection}
\subsubsection{\citeauthor*{ref_paper_28} \citeyearpar{ref_paper_28}}

The authors used machine learning for the detection of the arrhythmia from ECG signals. The author used a single lead recording over a long period of time. The authors achieved up to 92.7\% sensitivity and 86\% accuracy in the first test, while 95.7\% sensitivity and 75\% accuracy in the second test. The authors utilized a GPU for the calculations to reduce the amount of training and prediction time. The authors suggest that good feature selection and guidelines are necessary for efficient machine learning models.

\subsection{Automatic model selection for the optimization of SVM kernels}
\subsubsection{\citeauthor*{ref_paper_3} \citeyearpar{ref_paper_3}}

The authors used novel criteria for model selection against the custom SVM classifier. The system needs prior knowledge of machine learning technology. The system was able to handle the data up to two parameters independently. The system was unable to handle multi class data, hence it should be partitioned appropriately. The automation system reduced the probability of errors.

\subsection{Ranking learning algorithms: Using IBL and meta-learning on accuracy and time results}
\subsubsection{\citeauthor*{ref_paper_23} \citeyearpar{ref_paper_23}}

The authors used a meta ranking approach for the selection of machine learning algorithms instead of the baseline approach. The models are ranked according to the features of the data. The methodology used for the ranking is based on the success rate of the model and time parameter. The author suggested that the study will be beneficial to develop future ranking methods.

\subsection{A survey on the explainability of supervised machine learning}
\subsubsection{\citeauthor*{ref_paper_14} \citeyearpar{ref_paper_14}}

The authors conducted a survey of the machine learning system in finance, healthcare and various other industries. This survey provided the overview of supervised learning algorithms. The author suggests that the uncertainty about AI and machine learning is holding industrialization back. Also extremely complicated models are required for the complex industrial tasks making it harder to adopt. The survey suggested that the machine learning system was widely used during COVID-19 pandemic. Changing the approach towards the solution will increase the adaptation of machine learning. Also setting up clear ethical clauses in machine learning is necessary for wide use of this technology.

\subsection{Machine learning in medicine}
\subsubsection{\citeauthor*{ref_paper_24} \citeyearpar{ref_paper_24}}

The author commented on the lack of meaningful research of machine learning in the medical field. The authors studied both supervised and unsupervised machine learning algorithms. The author concluded that the supervised learning methods are beneficial in risk assessment and detection of known diseases, while the unsupervised methods are beneficial in the same task in case of the novel diseases. The author suggested that the large amount of data is necessary for accurate predictions. The subject specific methods as well as wider support is necessary for wide scale use of machine learning in the medical field.

\subsection{Machine learning in detection and classification of leukemia using smear blood images: a systematic review}
\subsubsection{\citeauthor*{ref_paper_15} \citeyearpar{ref_paper_15}}

The authors reviewed studies conducted to detect leukemia from blood smears with machine learning. The authors suggest that machine learning eased the load on the personnel. The machine learning systems achieved up to 97\% accuracy in leukemia detection in a case. Machine learning also produced more than 74\% accuracy in other cases.

\subsection{IoT for smart cities: Machine learning approaches in smart healthcare - A review}
\subsubsection{\citeauthor*{ref_paper_29} \citeyearpar{ref_paper_29}}

The authors used machine learning systems with IoT systems for smart cities. The authors suggested that sensors provide a large amount of data over time for the machine learning system. Authors suggest that machine learning will be extremely beneficial in the development of smart cities, advanced healthcare systems, and various other fields.

\subsection{Battling COVID-19 using machine learning: A review}
\subsubsection{\citeauthor*{ref_paper_20} \citeyearpar{ref_paper_20}}

The authors used a machine learning system for detection of viral disease. The machine learning system showed promising results. The authors are planning to do more research with various datasets to evaluate the efficiency and accuracy of machine learning.

\subsection{The role of machine learning algorithms for diagnosing diseases}
\subsubsection{\citeauthor*{ref_paper_11} \citeyearpar{ref_paper_11}}

The authors studied various machine learning techniques used for the diagnosis of diseases. The author suggests that machine learning reduces the number of human errors while making the system useful for daily life. The author suggests that KNN, SVM, DT, and RF models showed good results. The author also suggested that ensembling methods would produce even better results.

\subsection{Classification of pachychoroid disease on ultrawide-field indocyanine green angiography using auto-machine learning platform}
\subsubsection{\citeauthor*{ref_paper_8} \citeyearpar{ref_paper_8}}

The authors employed an auto machine learning platform to classify the pachychoroid disease. The authors used direct and indirect approaches for comparisons. The authors found that the indirect approach of equalizing the data produced better results. The author was satisfied with the auto machine learning algorithms, but suggested that the technology is still unstable for proper medical uses.

\subsection{Automatic model selection for support vector machines}
\subsubsection{\citeauthor*{ref_paper_2} \citeyearpar{ref_paper_2}}

The authors proposed an automation system to generate the most-suited SVM models. Theory behind the research was the approximation of loo stopping rate was reducing the efficiency of the SVM models. The author suggested that better loo stopping will reduce the training and testing time. The authors used stalag collection after scaling appropriately for the model generation. The produced model is suitable for data upto 1000 variables. The system is currently limited to SVM models and RBF kernels.

\subsection{Deep transfer learning for industrial automation: a review and discussion of new techniques for data-driven machine learning}
\subsubsection{\citeauthor*{ref_paper_13} \citeyearpar{ref_paper_13}}

The authors used transfer learning in the industrial tasks for industrial automation systems. The authors suggested that transfer learning and continuous learning technologies are important for rapid industrialization of machine learning technologies. These will be obtained by reducing the gap between research and practical uses of these technologies. Author also suggests that machine learning performs better in dynamic environments of industries.

\subsection{Benchmarking of machine learning for anomaly based intrusion detection systems in the CICIDS2017 dataset}
\subsubsection{\citeauthor*{ref_paper_21} \citeyearpar{ref_paper_21}}

The authors used machine learning for the detection of anomalies. The DT, KNN, SVM and ANN algorithms were used for the tasks. The author also used some unsupervised learning algorithms, but supervised learning algorithms provided better results. The DT and KNN models provided bet results. The author concluded that supervised machine learning is still limited in case of complex and novel problems, but otherwise produced better results than unsupervised learning algorithms.

\subsection{DDOS detection using machine learning technique}
\subsubsection{\citeauthor*{ref_paper_9} \citeyearpar{ref_paper_9}}

The authors used random forest methods for detection of DDOS attacks. Five properties of the DDOS attacks are used as the variable in machine learning algorithms. The authors suggested that the machine learning model random forest produced good results.

\subsection{Classification of arrhythmia using machine learning techniques}
\subsubsection{\citeauthor*{ref_paper_4} \citeyearpar{ref_paper_4}}

The authors used machine learning to automatically classify the cardiac arrhythmias in embedded systems. The models were focused on three factors: accuracy, predictions and ease-of-use of the system. The author suggested that the volume of data is indirectly proportional to the understanding of the data. The designed model produced satisfactory results, hence proving the usefulness of machine learning in the medical field.

\subsection{Auto machine learning-based modelling and prediction of excavation-induced tunnel displacement}
\subsubsection{\citeauthor*{ref_paper_1} \citeyearpar{ref_paper_1}}

Authors proposed an AutoML system for the prediction of excavation of displacement. Six generic models were provided with seven input points and two properties. The system was implemented with 10-fold cross-validation methods to increase the efficiency. The models produced efficient and precise predictions. The authors concluded that AutoML is optimal for these predictive tasks. Authors also suggested that GA-MLP also provided satisfactory results.

\subsection{Predicting reaction yields via supervised learning}
\subsubsection{\citeauthor*{ref_paper_10} \citeyearpar{ref_paper_10}}

The authors used a machine learning system in case with a large number of features. Authors suggested that a similar result to traditional statistical methods was achieved with machine learning algorithms. The author suggested that good data collection techniques are necessary for better results. The authors suggested that use of machine learning methods are successful in the chemical domain. The author also suggests that more development is necessary to accommodate the vast amount of features in the chemical field.

\section{Inference of Literature Review} \label{sec:inference _of_literature_review}

The following inference can be drawn from the basis of literature review

\vspace{-1.5em}

\begin{itemize}
  \item Most papers introduced here use supervised learning algorithms for the prediction. This technique needs a well labeled dataset for training and evaluation. Therefore, expertknowledge is needed during the training and evaluation process.
  \item Some papers concluded that it is possible to select suitable models with minimum human interactions automatically. The knowledge driven and data driven approaches are used for automated model suggestions.
  \item Most papers also showed that on-shelf models can be used for accurate and efficient use cases, but few authors used novel kernels and models to get greater accuracy and efficiency for predictions.
  \item The recent paper showed that there is great need for models that can be setup faster for required work. To solve this problem we can use limited data for training which is possible with supervised learning algorithms according to few studies.
  \item Finally it can be inferred that a very negligible amount of work has been done for the automated model selection for general prediction purposes. Moreover, a limited amount of study is done with the general population as end users. Hence, this system is focusing on training, evaluation and selection of models for users with limited knowledge of data science.
\end{itemize}
