\thispagestyle{fancy}
\chapter{Literature Survey} \label{ch:literature_survey}
\section*{\centering Chapter \thechapter}
\section*{\centering Literature Survey}
\nocite{*}

\section{Introduction}\label{sec:introduction_literature_review}

COVID-19 has become a global pandemic in the past few years. It targeted the human respiratory system and caused serious infection in the patients. This infection affected patients quite severely in the case of the present comorbidities. This resulted in a fatal infection in some cases. These comorbidities were mainly diabetes, cardiovascular disease, prior respiratory system damage, etc. Cardiovascular diseases, also known as CVDs, are caused by disorders in the cardiovascular system, including the heart and other blood-circulating organs. Major Cardiovascular diseases are life-threatening, while minor Cardiovascular diseases can lead to heart attacks or strokes. The presence of this disease can lead to severe COVID-19 infection.

Cardiovascular disease is generally accompanied by various symptoms, but arrhythmia is the most common symptom among them. \cite{ref_paper_covid_1} suggest that arrhythmia is one of the most common symptoms in COVID-19 patients. An arrhythmia is an irregular beating of the heart. This can be easily detected by the ECG report from the patients. The ECG graphs are usually recorded at a 128 Hz frequency over a few minutes. These graphs are studied by the medical staff for the detection of arrhythmia. If any irregularity is present, the patient is said to have arrhythmia. This process can be easily automated by the machine learning system.

\citeauthor{ref_paper_covid_1} also suggests that arrhythmia was present in about 7\% of total COVID-19 cases in Wuhan and 14.8\% of patients with poor outcomes had arrhythmia. The authors of \cite{ref_paper_covid_2,ref_paper_covid_3} carried out seperate surveys of 17 studies consists of 5815 patients are concluded similar findings that up to 9.3\% of COVID-19 confirmed patients had been detected with arrhythmia. The surveys \cite{ref_paper_covid_4, ref_paper_covid_5} showed that up to 94.4\% of patients with arrhythmia resulted in fatal infection and 95.8\% of patients with severe infections had arrhythmia. In the \cite{ref_paper_covid_6}, author suggests that only 8\% of patients with arrhythmia had prior cardiovascular disease, while in 56\% of the cases, arrhythmia symptoms were new onset after contracting COVID-19 disease.

\section{Machine Learning}\label{sec:machine_learning}

Machine learning is becoming a widespread technology. Machine learning is a technology where the computers learn to make decisions based on provided data. Machine learning uses statistical modeling and advanced mathematics to build decision making models.

The industries are making large amounts of data every day. Making sense of this data will give an advantage over the competitors. Machine learning technology is capable of this feat. Hence, many industries are already using machine learning. The medical industry is one of them. The medical industry uses machine learning for the research and detection of various diseases. Well trained machine learning models are extremely accurate and fast. This allows faster diagnosis, which leads to better patient care. Supervised machine learning and unsupervised learning are two main types of machine learning systems.

\subsection{Supervised Machine Learning}\label{subsec:supervised_machine_learning}

Supervised machine learning is a subset of machine learning technology. The supervised learning system uses the labeled data to derive meaningful results. The computer learns the patterns from the data and uses labels to increase the accuracy of the models. The supervised models become more efficient with large amounts of data. Supervised learning models are generally used for classification and regression problems.

\section{Review of related literature}\label{sec:review_of_related_literature}

\subsection{Auto machine learning-based modelling and prediction of excavation-induced tunnel displacement}
\subsubsection{\citeauthor*{ref_paper_1} \citeyearpar{ref_paper_1}}

The paper uses seven inputs and 2 properties.

\section{Inference of Literature Review} \label{sec:inference _of_literature_review}

The following inference can be drawn from the basis of literature review

\vspace{-1.5em}

\begin{itemize}
  \item Most papers introduced here use supervised learning algorithms for the prediction. This technique needs a well labeled dataset for training and evaluation. Therefore, expertknowledge is needed during the training and evaluation process.
  \item Some papers concluded that it is possible to select suitable models with minimum human interactions automatically. The knowledge driven and data driven approaches are used for automated model suggestions.
  \item Most papers also showed that on-shelf models can be used for accurate and efficient use cases, but few authors used novel kernels and models to get greater accuracy and efficiency for predictions.
  \item The recent paper showed that there is great need for models that can be setup faster for required work. To solve this problem we can use limited data for training which is possible with supervised learning algorithms according to few studies.
  \item Finally it can be inferred that a very negligible amount of work has been done for the automated model selection for general prediction purposes. Moreover, a limited amount of study is done with the general population as end users. Hence, this system is focusing on training, evaluation and selection of models for users with limited knowledge of data science.
\end{itemize}
