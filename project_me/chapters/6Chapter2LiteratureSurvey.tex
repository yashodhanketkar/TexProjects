\thispagestyle{fancy}
\chapter{Literature Survey} \label{ch:literature_survey}
\section*{\centering Chapter \thechapter}
\section*{\centering Literature Survey}
\nocite{*}

\section{Introduction} \label{sec:introduction_literature_survey}

Arrhythmia is one of the most common symptoms in COVID-19 patients
\cite{babapoor2020arrhythmia}. Arrhythmia was present in 7\% of total COVID-19 cases in Wuhan
and 14.8\% of patients with poor outcome had arrhythmia. Data from 17 studies consisting of
5815 patients suggested that up to 9.3\% of confirmed patients had been detected with
arrhythmia \cite{mulia2021atrial, liu2020clinical}. Up to 94.4\% of total patients died had
arrhythmia and 95.8\% of patients with serious infection had arrhythmia
\cite{beri2020cardiac,ren2020clinical}. Only 8\% of patients with arrhythmia had prior
cardiovascular diseases while in 56\% arrhythmic symptoms were new onset after contracting
COVID-19 disease according to study
\cite{babapoor2020arrhythmia, liu2020clinical, yarmohammadi2021frequency}.

\section{Methodologies Used By Authors} \label{sec:methodologies_used_by_authors}

In paper \cite{sun2020multi} the author utilized ensemble classifiers to detect anomalies in
ECG signals. This approach of combining multiple classifiers with each other for prediction
proved effective, as accuracy of ensemble classifiers was significantly greater than single
classifiers. This approach is also used by other researchers to improve prediction accuracy of
supervised learning models \cite{huang2020accurate,rajak2020applying,liu2020parallel}. The
studies also presented greater accuracy and validation score for anomaly detection
\cite{huang2020accurate,liu2020parallel}, while one study showed that accurate predictions can
be obtained without any prior information about the same class \cite{rajak2020applying}. The
standard classifiers can be used to obtain accurate and precise predictions for incoming data
streams automatically \cite{imbrea2021automated}.

For detection presence of epileptic seizure in patients, the authors preprocessed data by
extracting suitable features and applied supervised and unsupervised machine learning
classifiers to it. The predictions from test data suggest that supervised learning models are
more effective than unstructured learning models \cite{siddiqui2020review}. The standard
classifiers can be automated for accurate and precise prediction \cite{imbrea2021automated}.
Author of study \cite{jha2020cardiac}, also presented that the off-the-shelf classifier SVM is
very efficient for detection of arrhythmia in ECG signals.

The ensemble classifiers approach i.e. combination of multiple classifiers together for
predictions can be used for anomaly detection. The ensemble classifiers can be more effective 
and accurate than individual supervised learning models
\cite{sun2020multi, liu2020parallel, huang2020accurate}. The author of paper
\cite{rajak2020applying} discovered that accurate predictions can be obtained without prior
information about past target values of the same class.

The artificial neural networks are also very effective and accurate for arrhythmia detection
from ECG signals. The neural networks can process raw ECG signals and make predictions
\cite{hannun2019cardiologist}. Smaller scale neural networks are also very efficient in
detection processes \cite{sannino2018deep}. The neural network predictions are comparatively
more accurate and precise than well known classifiers.

The lightweight classifiers such as LDA classifier showed higher accuracy than SVM classifier
in low energy system \cite{chen2013design}. Self adapting learning algorithms are also
efficient for low energy lightweight systems \cite{lei2007afc, owis2002study}. The self
learning algorithms make the system more dynamic and adapt to incoming signals. This dynamic
system takes very little time for identification tasks, whereas classification tasks are more
difficult to manage and are very time consuming.

\section{Automated model selection approaches} \label{sec:automated_model_selection_approaches}

In the paper \cite{natasa2008}, the author used 2 years of dataset collected from glumo lake,
the use of expert knowledge is used for model training and selection. The mixed approach of
data driven and knowledge driven modeling is used for successful application. While author
of the paper \cite{lee2000automatic}, used loo rate and stopping criteria for model
selection. The author considered 8 different problems and discovered that larger loo rate is
more desirable. Author also suggested that modeling difficulties can only be found with careful
numerical calculations.

In paper \cite{malkomes2016bayesian}, novel kernels are used to obtain explanations of the
dataset. This approach leads to discovery of unseen models with minimal human interaction.
The author of the paper \cite{JSSv034i12}, used the glmulti package to build new models, these
models are unique and flexible. The models are optimized automatically and provide a
multi-model interface. With this approach a large set of models can be considered for selection
purposes at a rapid rate.

\section{Inference of Literature Review} \label{sec:inference _of_literature_review}

The following inference can be drawn from the basis of literature review

\vspace{-1.5em}

\begin{itemize}
    \item Most papers introduced here use supervised learning algorithms for the prediction.
    This technique needs a well labeled dataset for training and evaluation. Therefore, expert
    knowledge is needed during the training and evaluation process.
    \item Some papers concluded that it is possible to select suitable models with minimum
    human interactions automatically. The knowledge driven and data driven approaches are used
    for automated model suggestions.
    \item Most papers also showed that on-shelf models can be used for accurate and efficient
    use cases, but few authors used novel kernels and models to get greater accuracy and
    efficiency for predictions.
    \item The recent paper showed that there is great need for models that can be setup faster
    for required work. To solve this problem we can use limited data for training which is
    possible with supervised learning algorithms according to few studies.
    \item Finally it can be inferred that a very negligible amount of work has been done for
    the automated model selection for general prediction purposes. Moreover, a limited amount
    of study is done with the general population as end users. Hence, this system is focusing
    on training, evaluation and selection of models for users with limited knowledge of data
    science.
\end{itemize}
