\thispagestyle{fancy}
\chapter{Conclusion And Future Scope} \label{ch:conclusion_and_future_scope}
\section*{\centering Chapter \thechapter}
\section*{\centering Conclusion And Future Scope}

\section{Conclusion} \label{sec:conclusion}
As the number of COVID-19 are still increasing, the mortality rate is still rising. The mortality rate is higher in cases with comorbidities. Cardiovascular diseases are few of the comorbidities that drive this mortality rate. These diseases can be diagnosed earlier with a machine learning system. In this project we built an automated model training and selection system. This system can be used for predicting the presence of arrhythmia, which is the most common symptom in cardiovascular diseases.

The system uses supervised learning algorithms to generate models. This leads to efficient training and higher accuracy in predictions. The system can be fine tuned with by users, or it can be directly used without any formal training. The system showed good performance for similar datasets, so it can be used for general prediction purposes too.

\section{Future Scope} \label{sec:future_scope}
The current system only provides solutions to binary classification problems, but it can be used for multiclass classification problems. The provision of user created model templates can be done, with user provided performance modification. The system can be connected directly to the hospital servers to train models from patient records directly with the help of the RPA system \cite{ref_paper_self_rpa}.
